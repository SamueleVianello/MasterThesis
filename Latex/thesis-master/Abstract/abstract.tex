\chapter{Abstract}
\label{chpr:abstract}

Bitcoin is increasingly establishing itself as a digital investment asset other than a mere speculative instrument. In this work, we explore this subject by studying the correlation that Bitcoin has with other standard assets such as stock and bond indexes,  currencies and commodities. We first study the empirical correlation and its statistical significance, then calibrate more sophisticated asset models including jump diffusion and stochastic volatility in their multivariate generalizations to obtain the correlation matrices also under those frameworks. Results are closely related and Bitcoin correlation  with any other asset is confirmed to be very low. Given this result, we perform optimal portfolio allocation analyses to investigate its diversification properties, both through Markowitz mean-variance optimization and using the CVaR as the portfolio risk measure. Evidence shows that allocating a small percentage of wealth in the digital assets proves to be extremely beneficial in terms of lowering the risk and increasing the expected returns.


%Bitcoin si sta sempre più imponendo sul mercato come un asset d'investimento piuttosto che un semplice strumento speculativo. In questo lavoro,  analizziamo questo argomento tramite lo studio della correlazione dei rendimenti di Bitcoin con quelli di altre categorie di asset quali indici azionari e obbligazionari, valute e commodity. Per prima cosa studieremo le correlazioni empiriche e la loro significatività statistica, poi verranno calibrati modelli di  asset più sofisticati che includono volatilità stocastica e salti nella loro versione multivariata, in modo da ottenere le matrici di correlazione anche in questi framework. I risultati ottenuti sono molto simili tra i diversi approcci e la correlazione di Bitcoin con gli altri asset risulta molto bassa. Date queste risultanze, effetuiamo uno studio sull'allocazione di portafoglio ottimale per analizzare le proprietà di diversificazione di Bitcoin, sia attraverso l'ottimizzazione media-varianza alla Markowitz che utilizzando il CVaR come misura di rischio del portafoglio. Le evidenze numeriche mostrano che allocare una piccola percentuale della nostra ricchezza nell'asset digitale si dimostra estremamente vantaggioso sia in termini di riduzione del rischio che nell'aumentare i rendimenti attesi.


