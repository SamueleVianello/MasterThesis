\chapter{Calibration of the Models}
\label{chpr:calibration}

In this chapter we will explain how the different models were calibrated and what difficulties were overcome. Empirical results are included for each section.
In particular, we will calibrate the parameters under the physical measure though maximum likelihood estimation since we only have data for the time-series of the asset prices. 

\bigskip

\section{Maximum Likelihood Calibration}
\label{sec:calib_overview}
The calibration method that we implemented is the \textit{maximum likelihood} approach, which is a statistical method to obtain the parameters of a target distribution family that best approximate the unknown distribution of the observed data.

Let us call $X = \left\lbrace x_1, x_2, ... , x_N \right\rbrace$ the set of available observations and let $\psi= \left\lbrace \alpha_1, \alpha_2, ... \right\rbrace $ the set of parameters of the distribution $ \mathcal{D}$ that we want to calibrate.
Moreover, define $f_\mathcal{D} (x ; \psi)$ to be the probability density function (pdf) of distribution $\mathcal{D}$ with parameters $\psi$ computed at $x \in \mathbb{D}$, where $\mathbb{D}$ is the domain of the density function.

Our aim is to find the best set of parameters $\psi$ such that we can describe $X$ as samples taken from $\mathcal{D}$:
\begin{equation}
x_i \sim \mathcal{D} (\psi), \: i = 1, \dots, N.
\end{equation}

The estimation of the parameter set $\psi$ can then be obtained by:
\begin{equation}
	\hat{\psi} = \argmax_{\psi \in \Psi} \:\mathcal{L}(\psi |  X),
\end{equation}
\noindent
where $\mathcal{L}(\psi |  X)$ is the likelihood function for distribution $\mathcal{D}(\psi)$ given the observed data $X$.

In the case that the data in $X$ are i.i.d. we have that the likelihood function can be computed as the product of the probability density function computed at each observation $x_i$:

\begin{equation}
\label{eq:log_likelihood}
\mathcal{L}(\psi |  X) = \prod_{i=1}^{N} f_{ \mathcal{D}}(x_{i}; \psi).
\end{equation}

In practice, it is often common to consider the \textit{log-likelihood} function $\ell(\psi |  X)  = \log \mathcal{L}(\psi |  X)$. The natural logarithm is a monotonic function so the maximum of the log-likelihood function is achieved in the same point as the basic likelihood function.\footnote{By how the likelihood function is defined, it can attain only positive values so that the logarithm of  $\mathcal{L}(\psi |  X)$ is always well defined.}
Taking the logarithm of \eqref{eq:likelihood} also simplifies the expression in that the product now becomes a sum:
\begin{equation}
\label{eq:likelihood}
\ell (\psi |  X) =\mathcal{L}(\psi |  X) = \sum_{i=1}^{N} \log f_{ \mathcal{D}}(x_{i}; \psi).
\end{equation}

The issues that arise from the definition of a maximum likelihood estimator are mainly two.
The first problem is that to use equation \eqref{eq:likelihood} we need to know the pdf of the distribution: this might not always be the case with complicated models that only have an explicit expression for their characteristic function. This is what happens for Heston and Bates model, and, as we will see, we are going to need to invert the chf via Fourier inversion and obtain the pdf numerically.

The second issue is that we need to solve a maximization problem in order to obtain an estimate for the parameters of the distribution: it is well known how optimization routines might perform well for some special circumstance and badly under other, especially when the number of parameters increases. The trade-off is, as usual, between fast computations and robustness of the optimization. To solve this, we will opt for a combination of global and local optimizers.




\bigskip
\section{Calibration of Merton Model}
\label{sec:merton_cal}
Let us start by considering how we calibrated the jump diffusion model by Merton. 
We will first present how to obtain the parameters for a single asset model and then move to a multi-asset one.



\subsection{Single Asset Merton Calibration}
As we have already introduced, the calibration method that we use is the maximum likelihood approach.

We are going to calibrate the values of the 5 parameters $\psi= \left\{ \mu, \sigma, \mu_J, \sigma_J, \lambda \right\}$ of the jump diffusion model based on the observations of the log-returns $\Delta x_i = \log \frac{S_i}{S_{i-1}}$ for each time interval $\Delta t$. In our analysis, since we are using \textit{daily} log-returns, $\Delta t = 1/255$. 

Considering daily log-returns allows us to assume that the different samples of $\Delta x$ are independent and identically distributed according to the distribution of log-returns in Merton model given in \eqref{transitional}.
For ease of reading, we present it again here:
\begin{equation}
\label{eq:merton_full_pdf}
f_{\Delta x } (x; \psi) = \sum_{k=0}^{\infty} \mathbb{P}(N = k) f_{\Delta x | N = k}(x ; \psi) .
\end{equation}

The formula represents an infinite mixture of Gaussian distributions, due to the infinite possible realization of the Poisson variable that accounts for the arrival of jumps. 
This fact has a great downside in terms of maximum likelihood estimation. As discussed in \citep{HONORE1998}, the infinite Gaussian mixture causes the problem of maximizing the (log-)likelihood to be unbounded and thus intractable.

To solve this issue we can introduce a first order approximation, as it has been proposed in \citep{BALLTOROUS1983}. 
Since we are considering a small time interval $\Delta t $, only the first terms of the series in \eqref{eq:merton_full_pdf} become significant. 
In particular, we have that the probabilities to have $k = 0, 1, 2$ jumps in a single time step, as presented in \eqref{eq:pois_pdf}, are:
\begin{subequations}
	\begin{align}
	\mathbb{P}(N = 0) &= e^{-\lambda \Delta t}, \\
	\mathbb{P}(N = 1) &= \lambda \Delta t \: e^{-\lambda \Delta t}, \\
	\mathbb{P}(N = 2) &= \frac{(\lambda \Delta t)^2}{2} \: e^{-\lambda \Delta t},
	\end{align}
\end{subequations}
\noindent
since $N \sim Pois(\lambda \Delta t)$, which is the Poisson process counting jumps that happen in a $\Delta t$ interval.
Considering $\lambda \Delta t$ as small, we can approximate to the first order  $e^{-\lambda \Delta t} =  1 - \lambda \Delta t +  o((\lambda \Delta t)^2)$. 
We thus obtain that:
\begin{subequations}
	\begin{align}
	\mathbb{P}(N = 0) &= 1-\lambda \Delta t , \\
	\mathbb{P}(N = 1) &= \lambda \Delta t \: (1- \lambda \Delta t) = \lambda \Delta t + o((\lambda \Delta t)^2), \\
	\mathbb{P}(N = 2) &= \frac{(\lambda \Delta t)^2}{2} \: (1-\lambda \Delta t) = o((\lambda \Delta t)^2).
	\end{align}
\end{subequations}


The result is that the only relevant terms in \eqref{eq:merton_full_pdf} are the ones for $ k = 0, 1$.
The formula for the transition density thus  becomes:
\begin{equation}
f_{\Delta x} (x; \psi) = \mathbb{P}(N = 0) f_{\Delta X | N = 0}(x; \psi) + \mathbb{P}(N = 1) f_{\Delta x | N = 1}(x; \psi).
\end{equation}
We can then write this equation explicitly by considering the different distributions of the log-returns in the case of no jumps and a single jump:
\begin{equation}
\label{eq:merton_pdf}
f_{\Delta x} (x; \psi) = (1 - \lambda \Delta t) \;
f_{\mathcal{N}}\Big(x ; \widetilde{\mu}\, \Delta t, \sigma^2 \Delta t\Big) + (\lambda \Delta t)\; f_{\mathcal{N}}(x ; \widetilde{\mu}\, \Delta t + \theta, \sigma^2 \Delta t+\delta^2)
\end{equation}
where $f_{\mathcal{N}}(x ; \mu, \sigma^2)$ indicates the pdf of a Gaussian with parameters $\mathcal{N}(\mu, \sigma^2)$ compute at $x$. We used $\widetilde{\mu} = \mu - \sigma^2/2 -\lambda \mu_J$ as a simpler notation to indicate the drift term in the return dynamics.

The distribution of the sample log-returns is thus given by equation \eqref{eq:merton_pdf} and we can plug it into \eqref{eq:log_likelihood} to obtain the log-likelihood function that we are going to maximize:
\begin{equation}
\label{eq:merton_loglikelihood_single}
	\ell(\psi |  \Delta x) = \sum_{i=1}^{N} \log f_{ \Delta x}(\Delta x_{i}; \psi).
\end{equation}

One can then proceed to maximize equation \eqref{eq:merton_loglikelihood_single} to obtain the optimal parameter set $\psi$:
\begin{equation}
\hat{\psi} = \argmax_{\psi \in \Psi} \:\ell(\psi |  \Delta x),
\end{equation}
\begin{equation}
\Psi = \{ (\mu, \sigma, \mu_J, \sigma_J, \lambda) \in \mathbb{R}^5 \: |\: \sigma,\sigma_J, \lambda >0\}.
\end{equation}

\subsection{Multi-asset Merton Calibration}
\label{subsec:multi_merton_cal}

We now know  how to calibrate the Merton parameters when the underlying asset is one.
In the case of multiple assets, we decided to adapt \citep{PARSIMONIOUS2011} both for the model definition and the calibration procedure.
In the referenced paper, the authors present a \textit{parsimonious approach} to the extension of Heston Model to a multi-asset framework. Since this is the approach that we will follow for the multivariate Heston calibration and for Bates as well, we decided to also perform in the same way the calibration of the parameters for Merton.
Using a common approach will allow us to better compare the results, especially in terms of the correlation structure that is what we are ultimately interested in.

The first step in the calibration algorithm is to obtain the parameters of all the single asset models that we are studying. The general way to do this was explained in the previous section. 

The next step involves computing the correlation matrix between the different Brownian motions that drive each single-asset model. 
The way this computation is carried out in \citep{PARSIMONIOUS2011} is by asset pairs: we consider each possible couple of assets and try to obtain the model correlation that best approximates the observed correlation.
Let us see how this is achieved more in detail in the case of only two assets.

The problem we are trying to solve is, given the observed sample correlation $\rho_{obs}$ between two assets' returns, find the best model correlation $\rho_{model}$ that solves $\bar{\rho}(\rho_{model}) = \rho_{obs}$.
$\bar{\rho}$ is the expectation of the correlation that we find when computing the  empirical correlation of the returns simulated using $\rho_{model}$.

More precisely, we have to simulate $N_{sim}$ different realisations of the 2-asset model in a given time period $[0, T]$ with the same $\Delta t$ that we have in the observed data, using $\rho_{model}$ as the correlation between the two driving Brownian motions. We then proceed to compute the correlation of the simulated log-returns $\rho_{scen}$ in each scenario and then we take the average of this values. What we obtain is $\bar{\rho} = \mathbb{E}[\rho_{scen}] $.

In order to simulate a two-asset Merton model, we used the following discretization for the log-returns:
\begin{subequations}
	\label{eq:merton_discretization}
	\begin{align}
	x_i^{(1)} &= x_{i-1}^{(1)} + (\mu_1 - \sigma_1^2/2 -\lambda_1 {\mu_J}_1) \Delta t + \sigma_1 \sqrt{\Delta t} \;  z_1 + N_1 {y}_1 ,\\
	x_i^{(2)} &= x_{i-1}^{(2)} + (\mu_2 - \sigma_2^2/2 -\lambda_2 {\mu_J}_2) \Delta t + \sigma_2 \sqrt{\Delta t} \; z_2 + N_2 {y}_2 ,
	\end{align}
\end{subequations} 
where:
\begin{itemize}
	\item $i = 1, \dots , N_{step}$ represents the i-th time-step in our simulation: $x_i = x_{t_i}$ and $N_{step} = T / \Delta t$ so that $t_i = i \Delta t$. The value of the initial return $x_{t_0}$ does not affects the final correlation so we can simply set it to zero.
	\item $z_1$ and $z_2$ are the first and second component of a Gaussian vector $\mathbf{z} = (z_1,z_2)^T$ that is distributed as $\mathbf{z} \sim \mathcal{N}_2(\mathbf{0}, Corr)$. $Corr$ is the $2 \times 2$ matrix composed of ones in the main diagonal and $\rho_{model}$ in the remaining two spots. 
	More simply, $\mathbf{z}$ is a two dimensional Gaussian with zero mean with covariance matrix equal to the correlation matrix of the model.
	\item $N_1$ and $N_2$ are in general the realizations of two Poisson random variables with parameters $\lambda_1 \Delta t$ and $\lambda_2 \Delta t$ respectively. Given our first order approximation in the density and considering small $\Delta t$,  $N_1$ and $N_2$ are actually two Bernoulli with probability of success (i.e. a jump happens) equal to $\lambda_1 \Delta t$ and $\lambda_2 \Delta t$.
	\item Finally, $y_1$ and $y_2$ are the jump intensities and are realisations of Gaussian variables with parameters $\mathcal{N}({\mu_J}_1, {\sigma_J}_1^2)$ and $\mathcal{N}({\mu_J}_2, {\sigma_J}_2^2)$ . 
\end{itemize}

Through \eqref{eq:merton_discretization} we can thus simulate $ x_i^{(1,k)}$ and 
$ x_i^{(2,k)}$ for $k= 1, ... ,N_{sim}$  and compute for each scenario the correlation observed from the simulated returns $\rho_{scen}^{(k)} = corr(x_i^{(1,k)}, x_i^{(2,k)})$.
From these values we can easily compute $\bar{\rho}$:
\begin{equation}
\label{eq:expected_model_correlation}
	\bar{\rho} = \mathbb{E} [\rho_{scen}] = \frac{1 }{N_{sim}} \sum_{k=1}^{N_{sim}} \rho_{scen}^{(k)}.
\end{equation}


We have that  the value of the \textit{expected model correlation} $\bar{\rho}$ is  implicitly a function of the model correlation $\rho_{model}$. The equation we need to solve is then represented by:
\begin{equation}
\label{eq:expected_observed}
\bar{\rho}(\rho_{model}) = \rho_{obs},
\end{equation}
\noindent
as we have already stated earlier in this section.

\bigskip
In the case of an $n$-asset model, we have to repeat the previous passages for all possible $n(n-1)/2 $ different pairs of assets. We then can build a $n \times n$ matrix $M$ that has ones in the main diagonal and in each element $(l,m), l \neq m$ stores the corresponding model correlation $\rho_{model}^{(l,m)}$.
Since we are calibrating each value individually, this matrix may not be a well defined correlation matrix: it may happen that $M$ is not positive definite.
The last step to obtain a proper correlation matrix for our $n$-asset model is to perform some form of \textit{regularization} on $M$ that transforms it into a well defined correlation matrix: positive definite, with elements in the range $[-1,1]$ and ones in the main diagonal.

Through empirical study performed in the reference paper \citep{PARSIMONIOUS2011}, it turns out that the best algorithm between the one proposed by the authors is the regularization by J\"ackel, that we briefly present here.

\bigskip
\textbf{Regularization algorithm by J\"ackel:}
\begin{enumerate}
	\item[\textbf{Input}] Model correlation matrix $M$ (not necessarily positive definite but with ones on the main diagonal).
	\item Perform an eigenvalue decomposition on $M  = S \Lambda S^T$, where $\Lambda = diag(\lambda_l)$ and $\lambda_l$ are the eigenvalues of matrix $M$.
	\item Define $\Lambda^{'} = diag(\lambda_l^{'}$ with $\lambda_l^{'} = \max(\lambda_l, 0))$ as the diagonal matrix that contains the positive part of each eigenvalue.
	\item Create the diagonal matrix $T = diag(t_l)$ where $t_l =\big(\sum_{m} (S_{l,m})^2 \lambda_l^{'}\big)^{-1}$ .
	\item Define $B = \sqrt{T} S \sqrt{\Lambda^{'}}$.
	\item Set $\widehat{M} = B B^T$.
	\item[\textbf{Output}] Positive definite correlation matrix $\widehat{M}$.
\end{enumerate}

Hence, $\widehat{M}$ is the model correlation matrix that best represents the observed correlation values and that forms the structure of dependence between the Brownian motions that drive each asset in a multivariate Merton model.


\bigskip
\section{Calibration of Heston Model}
\label{sec:heston_cal}
Calibrating the Heston model on time-series data presents an additional difficulty since we only have data on the price process, while the model also takes into consideration the dynamics of the variance as a stochastic process. 

The problem of deducing the parameter of a \textit{non-observable} process from the observable states of a system is called \textit{filtering} and is in general a complicated numerical procedure. There are nonetheless some studies in the literature on the subject of calibrating the Heston model through filtering, for example \citep{filtering}.

In our analysis we preferred to keep the same approach to the calibration for all the three models. We will thus proceed to explain how we can perform a maximum likelihood calibration in the case of stochastic volatility and, in the next section, of both jumps and stochastic volatility.

\subsection{Single Asset Heston Calibration}
The first issue that we have to overcome is that an explicit formula for the probability density function of the log-returns is not available and we have to resort to performing a Fourier inversion on the characteristic function, as we already showed in equation \eqref{eq:chf_inversion}

This allows us to have a pdf \textcolor{red}{AGGIUNGERE GLOSSARIO }for the log-return $x_t = \log S_t $ but in order to be able to perform a maximum likelihood calibration in the same way that we did in the  previous section, we need to have the distribution for $\Delta x_t = \log (S_{t + \Delta t} / S_t)$ and make sure that they are i.i.d. so that we can apply \eqref{eq:log_likelihood}.

In order to obtain the pdf for the incremental log-returns $\Delta x_t$, we first need the distribution of  $\Delta x_{[0, t] }= \log (S_t / S_0)$. We can simply proceed as we explained in Chapter \ref{chpr:models} in the section about Heston model to get the characteristic function for $\Delta x_{[0, t] }$ conditioned on the values of $V_0$: 

\begin{equation}
\phi_{\Delta x_{[0, t] }}(u|V_0) =  \exp\{A(t,u) + C(t,u) V_0\}.
\end{equation}

The full expressions for $A(t,u)$ and $C(t,u)$ are given in \eqref{eq:heston_chf+ABC}.
Notice that we omitted the dependence on the model parameters $\psi = \{\mu, \kappa, \theta, \sigma_V, \rho \}$ for ease of notation.

\bigskip
We are now left with the distribution of $\Delta x_{[0, t] }$ as a function of $V_0$: this represents an issue since moving from $\Delta x_{[0, t] }$ to $\Delta x_t  = \Delta x_{[t, t + \Delta t]}$ we have a dependence on the value of $V_t$ which is not available. Moreover, even if data for $V_t$ was indeed available, considering different time-steps $t_i$ for $\Delta x_{t}$, the data samples of $\Delta x_{t_i}$ would not be identically distributed as they depend on levels of the variance $V_{t_i}$ which are different.

Hence the need for an \textit{unconditional} characteristic function to get rid of the dependence on initial variance. We followed the approach that was introduced in \citep{DRAGULESCU2002} of integrating out the dependence on $V_0$. The computations to obtain it were presented in Chapter \ref{chpr:models}.
The unconditional chf for the Heston model has the following expression:
\begin{equation}
 \phi_{\Delta x_t}(u) = \exp\{A(\Delta t,u) \} M_{\Gamma}(C(\Delta t,u))
\end{equation}
\noindent
with $M_{\Gamma}$ as the moment generating function of a Gamma distribution:
\begin{subequations}
\begin{align}
	M_{\Gamma} (z) &= \Big(\frac{\omega}{\omega-z}\Big)^\nu, \nonumber \\
	\omega&= \frac{2\kappa}{\sigma_V^2} ,\nonumber\\
	\nu&= \frac{2\kappa\theta}{\sigma_V^2}.\nonumber
\end{align}
\end{subequations}

We now have a specific expression for the unconditional characteristic function of $\Delta x_t$ and we can obtain the corresponding density by Fourier inversion:

\begin{equation}
\label{eq:uncond_inversion}
f_{\Delta x_t}(x) = \frac{1}{2\pi}\int_{-\infty}^{+\infty}  \phi_{\Delta x_t}(i u) e^{i u x} du
\end{equation}


Another advantage of using the uncoditional \acrfull{chf} is that since all the observed log-returns $\Delta x_i = \Delta x_{t_i} $ are considered to be sampled from the same distribution, we can numerically perform the inversion in \eqref{eq:uncond_inversion} using the Fast Fourier Transform (FFT). This allows for a great increase in the speed of the numerical computations since we can obtain the value of the pdf for all different $\Delta x_i$ with a single inversion and one interpolation.
The main ideas behind the FFT algorithm are explained in Appendix \ref{app:FFT}.


Now that we have the density function of $\Delta x$ with parameters $\psi = \{\mu, \kappa, \theta, \sigma_V, \rho \}$ we are able to compute the log-likelihood function by substituting \eqref{eq:uncond_inversion} in \eqref{eq:likelihood}.


In order to perform the maximization of the log-lokelihood function to calibrate the parameters, we have to make sure that the Feller condition \eqref{eq:feller_condition} is satisfied.

The MLE procedure will thus need to be performed including an additional constraint in the optimization given by Feller condition:
\begin{equation}
\widehat{\psi} = \argmax_{\psi \in \Psi} \:\ell(\psi |  \Delta x),
\end{equation}
\begin{equation}
	\Psi = \{ (\mu, \kappa, \theta, \sigma_V, \rho) \in \mathbb{R}^5 \: |\: \kappa,\theta,\sigma_V >0, \rho \in [-1,1], 2\kappa\theta \geq \sigma_V \}.
\end{equation}



\subsection{Multi-asset Heston Calibration}
\label{sec:multi_heston_cal}
As stated earlier in this paper, the approach to extending the Heston model to $N$ assets that we are taking into consideration is the parsimonious multi-asset model introduced by Dimitroff et al. in \citep{PARSIMONIOUS2011}.
It's called \textit{parsimonious} since, as we explained in chapter \ref{chpr:models}, we are only modelling the asset-asset correlations.

The procedure to calibrating an $N$ asset model is the same that we presented in the multi-asset Merton section and consists of three parts.

Firstly, we have to calibrate all the single asset parameters by themselves. Then we can proceed to compute each asset-asset correlation by solving \eqref{eq:expected_observed} through the simulation of the realization of the log-returns in different scenarios.
Finally, we perform J\"ackel regularization algorithm to make sure that the final output matrix is indeed a correlation matrix.

The main steps were all presented in section \ref{subsec:multi_merton_cal}, here we will only give the details for the simulation of a two-asset Heston model.

Since we now have two processes for each asset, we have to simulate the path for both, in each time-step.
The discretized dynamics of the log-returns and the variance for a single asset are the following:

\begin{subequations}
	\label{eq:discrete_heston}
	\begin{align}
		x_i &= x_{i-1} + (\mu -  \frac{V_{i-1}}{2})\Delta t + \sqrt{V_{i-1} \Delta t} \:z_x, \\
		V_i &= V_{i-1} + \kappa(\theta - V_{i-1} )\Delta t + \sigma_V \sqrt{V_{i-1} \Delta t} \: z_V,
	\end{align}
\end{subequations}
with $z_x$ and $z_V$ that are the two components of a Gaussian vector with mean zero and correlation $\rho$.\footnote{In a more extensive form: $ \mathbf{z} = \begin{pmatrix}
	z_x\\ z_V
	\end{pmatrix} \sim \mathcal{N}_2 \Big(\begin{pmatrix}
	0\\ 0\end{pmatrix}, \begin{pmatrix}
	1& \rho\\ \rho&1
	\end{pmatrix} \Big)$}

The issue that arises when moving from a continuous dynamics to a discrete one as in \eqref{eq:discrete_heston} is that the variance $V_i$ might assume negative values even if the Feller condition is verified. This happens because the second and third term in the equation of the variance process may be so negative that combined with the first term $V_{i-1}$ the result is less than zero.
Having $V_i < 0$ causes the next update at $i+1$ to be undefined since we would have to compute the square root of a negative value in a real framework.

To solve this issue, a number of possible solutions have been proposed and are collected in \citep{LORD2010}. Among those we have:
\textit{absorption}, which consists in setting $V_i = 0$ when it is negative, \textit{reflection}, taking the absolute value $V_i = |V_i|$, and finally \textit{full truncation}, which is the solution proposed  by the authors in \citep{LORD2010} and the one we  implement. It is obtained modifying \eqref{eq:discrete_heston} as follows:

\begin{subequations}
	\label{eq:full_truncation}
	\begin{align}
	x_i &= x_{i-1} + (\mu -  \frac{V_{i-1}^+}{2})\Delta t + \sqrt{V_{i-1}^+ \Delta t} \:z_x, \\
	V_i &= V_{i-1} + \kappa(\theta - V_{i-1}^+ )\Delta t + \sigma_V \sqrt{V_{i-1}^+ \Delta t} \: z_V.
	\end{align}
\end{subequations}
The notation $y^+ = \max(y, 0)$ indicates the positive part of $y$. 

Equations in \eqref{eq:full_truncation} represent the single asset simulation scheme but in order to calibrate the asset-asset correlations we need to simulate the time-series for a pair of assets.

The updating computations only amount to repeating the single scheme twice:
\begin{subequations}
	\label{eq:full_truncation2}
	\begin{align}
	x_i^{(1)} &= x_{i-1}^{(1)} + (\mu_1 -  \frac{\Big(V_{i-1}^{(1)}\Big)^+}{2})\Delta t + \sqrt{\Big(V_{i-1}^{(1)}\Big)^+ \Delta t} \:z_{x^{(1)}}, \\
	V_i^{(1)} &= V_{i-1}^{(1)} + \kappa_1(\theta_1 - \Big(V_{i-1}^{(1)}\Big)^+ )\Delta t + \sigma_{V^{(1)}} \sqrt{\Big(V_{i-1}^{(1)}\Big)^+ \Delta t} \: z_{V^(1)},\\
	x_i^{(2)} &= x_{i-1}^{(2)} + (\mu_2 -  \frac{\Big(V_{i-1}^{(2)}\Big)^+}{2})\Delta t + \sqrt{\Big(V_{i-1}^{(2)}\Big)^+\Delta t} \:z_{x^{(2)}}, \\
	V_i^{(2)} &= V_{i-1}^{(2)} + \kappa_2(\theta_2 - \Big(V_{i-1}^{(2)}\Big)^+ )\Delta t + \sigma_{V^{(2)}} \sqrt{\Big(V_{i-1}^{(2)}\Big)^+ \Delta t} \: z_{V^{(2)}}.
	\end{align}
\end{subequations}

The important difference with the single asset case is the correlation structure. In order to simulate the random vector $\mathbf{z} = (z_{x^{(1)}}, z_{V^{(1)}}, z_{x^{(2)}}, z_{V^{(2)}})^T$ we have to extract it from a 4-dimensional Gaussian with zero mean and covariance given by $\Sigma$ where:
\begin{equation}
\label{eq:corr_matrix}
	\Sigma = \begin{pmatrix}
	1 	& \rho_1 & \rho_{1,2} & \rho_{1,2} \rho_2\\
	\rho_1 & 1 & \rho_1 \rho_{1,2} & \rho_1 \rho_{1,2} \rho_2\\
	 \rho_{1,2} & \rho_1 \rho_{1,2}  & 1 & \rho_2 \\
	 \rho_{1,2} \rho_2 & \rho_1 \rho_{1,2} \rho_2&\rho_2  & 1
	\end{pmatrix}.
\end{equation}

We could also use the Cholesky decomposition of $\Sigma = L L^T$ in order to obtain $\mathbf{z} = L \mathbf{\tilde{z}}$ from a standard 4-dimensional Gaussian vector\footnote{The standard Gaussian vector of dimension $n$ is a normal random vector with mean vector zero and the identity matrix as covariance matrix: $\mathcal{N}_n (\mathbf{0}, \mathbf{I}_n)$. This means that all the components are independent from one another.} $\mathbf{\tilde{z}}$. In this case matrix $L$ would be lower triangular and defined as:
\begin{equation}
	L = \begin{pmatrix}
	1&0&0&0\\
	\rho_1 & \sqrt{1- \rho_1^2} &0&0\\
	\rho_{1,2} &0&\sqrt{1-\rho_{1,2}^2}&0\\
	\rho_{1,2}\rho_2 & 0& \rho_2 \sqrt{1-\rho_{1,2}^2}&\sqrt{1- \rho_2^2} 
	\end{pmatrix}.
\end{equation}

From here on, the procedure is the same as for Merton:
we need to solve \eqref{eq:expected_observed} for every possible asset pair to obtain the model correlation M.

The final step is to apply  J\"ackel regularization algorithm to obtain a valid correlation matrix $\widehat{M}$.

\bigskip
\section{Calibration of Bates Model}
\label{sec:bates_cal}
The algorithm to compute the value for the parameters of the Bates model is the same as the two previous ones, so in the following sections we will only focus on the main differences in the functions and equations.

\subsection{Single Asset Bates Calibration}
As it was the case for the Heston model, there is no explicit formula for the density of the log-returns $\Delta x_t$. Moreover, the characteristic function is dependent on the initial value of the variance process, unless we use the approximation that we presented in Chapter \ref{chpr:models} to obtain \eqref{eq:bates_uncond_chf}.

We can then go ahead and compute the corresponding density function by Fourier inversion: we can perform this computation using the FFT algorithm that allows for an increase in the performance speed. 
Arguing in the same way as in previous sections, by having the pdf of $\Delta x$ we can compute it for all the observations of $\Delta x_i$ with a single inversion.

Finally, to calibrate the model we need to perform the maximization of the log-likelihood function, making sure the parameters satisfy the Feller condition:

\begin{equation}
\widehat{\psi} = \argmax_{\psi \in \Psi} \:\ell(\psi |  \Delta x),
\end{equation}
\begin{multline}
\Psi = \{ (\mu, \kappa, \theta, \sigma_V, \rho, \mu_J, \sigma_J, \lambda) \in \mathbb{R}^8 \: |\: \\
\kappa,\theta,\sigma_V,\sigma_J, \lambda >0, \rho \in [-1,1], 2\kappa\theta \geq \sigma_V \}.
\end{multline}

\subsection{Multi-asset Bates Calibration}
Again, the generalization to the multivariate case we present is an adaptation from \citep{PARSIMONIOUS2011}. The model is be the same as the one in section \ref{sec:multi_heston_cal} with the addition of the jump component.

The calibration algorithm is composed of the three parts concerning the calibration of all the single asset models, the  computation of the asset-asset correlations by simulating the processes, and the regularization of the model correlation matrix in order to obtain an acceptable correlation matrix, with ones in the main diagonal, all other elements in the $[-1,1]$ range and that is positive definite.

The simulation for a single asset is performed through the following scheme, implementing the \textit{full truncation} approach to solving the issue with the variance being negative:

\begin{subequations}
	\label{eq:bates_sim}
	\begin{align}
	x_i &= x_{i-1} + (\mu -  \frac{V_{i-1}^+}{2}- \lambda \mu_J)\Delta t + \sqrt{V_{i-1}^+ \Delta t} \:z_x + N y, \\
	V_i &= V_{i-1} + \kappa(\theta - V_{i-1}^+ )\Delta t + \sigma_V \sqrt{V_{i-1}^+ \Delta t} \: z_V,
	\end{align}
\end{subequations}

where $y$ is the jump intensity and is distributed as a Gaussian $\mathcal{N}(\mu_J, \sigma_J^2)$ and $N$ is the Poisson variable with parameter $\lambda \Delta t$ which, given our approximation for small $\Delta t$, is equivalent to a Bernoulli with probability of success equal to $\lambda \Delta t$. 
$z_x$ and $z_V$ are the two components of a Gaussian vector with mean zero and correlation $\rho$.

The same scheme applied to a  2-asset model is the following:
\begin{subequations}
	\label{eq:bates_full_truncation2}
	\begin{align}
	x_i^{(1)} &= x_{i-1}^{(1)} + (\mu_1 -  \frac{\Big(V_{i-1}^{(1)}\Big)^+}{2} \lambda_1\mu_{J_1})\Delta t + \sqrt{\Big(V_{i-1}^{(1)}\Big)^+ \Delta t} \:z_{x^{(1)}} + N_1 y_1, \\
	V_i^{(1)} &= V_{i-1}^{(1)} + \kappa_1(\theta_1 - \Big(V_{i-1}^{(1)}\Big)^+ )\Delta t + \sigma_{V^{(1)}} \sqrt{\Big(V_{i-1}^{(1)}\Big)^+ \Delta t} \: z_{V^(1)},\\
	x_i^{(2)} &= x_{i-1}^{(2)} + (\mu_2 -  \frac{\Big(V_{i-1}^{(2)}\Big)^+}{2} \lambda_2 \mu_{J_2})\Delta t + \sqrt{\Big(V_{i-1}^{(2)}\Big)^+\Delta t} \:z_{x^{(2)}} + N_2 y_2, \\
	V_i^{(2)} &= V_{i-1}^{(2)} + \kappa_2(\theta_2 - \Big(V_{i-1}^{(2)}\Big)^+ )\Delta t + \sigma_{V^{(2)}} \sqrt{\Big(V_{i-1}^{(2)}\Big)^+ \Delta t} \: z_{V^{(2)}},
	\end{align}
\end{subequations}
where the parameters and variables have the same meaning as in the single asset framework.
The random vector $\mathbf{z} = (z_{x^{(1)}}, z_{V^{(1)}}, z_{x^{(2)}}, z_{V^{(2)}})^T$ is a 4-dimensional Gaussian with zero mean and covariance $\Sigma$ expressed in \eqref{eq:corr_matrix}. We can also consider the Cholesky decomposition $\Sigma= L L^T$ to sample the vector $\mathbf{z}$ from the independent Gaussian vector $\mathbf{\tilde{z}}$ by doing $\mathbf{z} = L \mathbf{\tilde{z}}$.

After we have calibrated each asset-asset correlation by solving \eqref{eq:expected_observed}, we need to perform J\"ackel regularization algorithm to obtain a valid model correlation matrix.

\bigskip
\section{Empirical results}
\label{sec:results_cal}
We now present the specifics of how the calibration procedure was implemented in our study, including the numerical results that it yielded.
All the code to perform the computations was developed in R for greater portability across platforms. 	\textcolor{red}{aggiungere recap sui dati}

\subsection{Merton Calibration Results}
One of the core aims of this study is to compare the values of the empirical correlation of Bitcoin with other main assets as presented in Chapter \ref{chpr:corr_analysis} with the same correlations computed using more sophisticated models, in our case Merton, Heston and Bates.

The calibration algorithm to compute the parameters of the Merton model was implemented step by step as presented in the previous sections.

In order to perform the single-asset calibration we need to specify the boundaries for the domain of the parameter set $\psi =  \left\{ \mu, \sigma, \mu_J, \sigma_J, \lambda \right\}$ and which optimizer we are going to use.
In our analysis, the optimization is carried out by first taking advantage of a global optimizer and then performing a local maximization\footnote{Since most optimization packages implemented in R usually perform minimization as default, we coded the minimization of the \textit{negative} log-likelihood instead of the maximization of the (positive) log-lkelihood.}

The global optimizer we adopted is implemented in the R package \texttt{DEoptim} and performs \textit{Differential Evolution} optimization: the DE algorithm starts with a population of points randomly sampled from the search space and at each iterations creates new candidate solutions and only keeps the one that best improve the objective function. More information on the package can be found in \citep{DEoptim_manual} and an example of its usage in calibrating a JD process in \citep{DEoptim_jumpdiffusion}. The general DE algorithm was first introduced in \citep{DEoptim_book}.

The local optimizer we used is instead implemented in the function \texttt{nlminb} and it performs a quasi-Newton optimization with box constraints. The function is included in the \texttt{stats} package which is native in most versions of R.

\bigskip
The maximization of the log-likelihood function is notoriously a difficult task since its shape has many different maxima and most of them have similar values for the function. It is thus necessary to restrict the search domain to values that make both numerical and financial sense.
For the calibration of the parameters in the Merton model, we used the  boundaries reported in the following table:
\bigskip

\begin{center}
	\begin{tabular}{lccccc}
		
		&$\mu$ & $\sigma$ & $\mu_J$ & $\sigma_J$ & $\lambda$ \\
		\midrule
		lower bound & -5 & $10^{-5}$ & $2*q_{0.1\%}(r_{sample})$ &$10^{-4}$ &$10^{-5}$\\
		upper bound & 5 & 2 & $q_{0.5\%}(r_{sample})$ & 0.1& 10\\
		\midrule
	\end{tabular}
\end{center}


\bigskip
We choose to use a different boundary on each asset for the jump mean $\mu_J$  because we aim to consider as jumps only true shocks: that is why we base our boundaries on the quantiles $q_{\alpha}(r_{sample})$ of the daily log-return distribution of our sampled data. 
Moreover, we limit the volatility of the jumps to 0.1 in order the have  a jump distribution that is more concentrated around the value of $\mu_J$.
This way, we also avoid having extremely high values of $\lambda$, which instead is the case in \citep{HONORE1998}. 
For our assets, the values of the boundaries for $\mu_J$ are reported in Table \ref{tab:mu_J_bounds}.


\begin{table}
	\small
	\centering
	\caption[Upper and lower bounds for jump mean in Merton]{Upper and lower bounds for the parameter $\mu_J$.}
\begin{tabular}{lll}
	& lower bound & upper bound \\
		\midrule
	btc & -0.9316 & -0.2541 \\
	bric & -0.1085 & -0.0342 \\
	sp500 & -0.0911 & -0.0324 \\
	eurostoxx & -0.1350 & -0.0448 \\
	nasdaq & -0.1020 & -0.0363 \\
	bond\_europe & -0.0413 & -0.0148 \\
	bond\_us & -0.0163 & -0.0058 \\
	bond\_eur & -0.0380 & -0.0160 \\
	eur & -0.0372 & -0.0166 \\
	gbp & -0.0402 & -0.0148 \\
	chf & -0.0445 & -0.0163 \\
	jpy & -0.0576 & -0.0196 \\
	gold & -0.1108 & -0.0343 \\
	wti & -0.1776 & -0.0608 \\
	grain & -0.1143 & -0.0462 \\
	metal & -0.1076 & -0.0344 \\
	\midrule
	
\end{tabular}

\label{tab:mu_J_bounds}
\end{table}



The results for all the single asset calibration are presented in Table \ref{tab:merton_params}. We can see that for some assets the calibration procedure yields values of $\mu_J$ and $ \sigma_{J}$ that are on the boundaries: this is because the model is trying to explain most of the daily variations as jumps and setting boundaries avoids this phenomenon.

The resulting distribution for the log-returns obtained from the calibration is graphically  compared to the empirical distribution of the samples for each asset in Appendix \ref{app:hist}.
\bigskip


\begin{table}
	\small
	\centering
	\caption[Merton calibrated parameters]{Numerical results for the calibration of the parameters of the Merton model for each asset.}
	\begin{tabular}{lrrrrr}
		& $\mu$ & $\sigma$ & $\mu_{J}$ & $\sigma_{J}$ & $\lambda$ \\
		\midrule
		btc & 1.791 & 0.9591 & -0.315 & 0.1000 & 2.152 \\
		bric & 0.001 & 0.1628 & -0.034 & 0.1000 & 3.106 \\
		sp500 & 0.109 & 0.1245 & -0.032 & 0.0928 & 5.196 \\
		eurostoxx & 0.025 & 0.1985 & -0.045 & 0.1000 & 5.577 \\
		nasdaq & 0.146 & 0.1455 & -0.036 & 0.0865 & 4.935 \\
		bond\_europe & 0.026 & 0.0784 & -0.015 & 0.0015 & 3.027 \\
		bond\_us & 0.024 & 0.0300 & -0.006 & 0.0001 & 3.778 \\
		bond\_eur & 0.025 & 0.0836 & -0.016 & 0.0012 & 3.085 \\
		eur & -0.010 & 0.0819 & -0.017 & 0.0001 & 2.766 \\
		gbp & -0.013 & 0.0783 & -0.040 & 0.1000 & 0.308 \\
		chf & -0.018 & 0.0877 & -0.016 & 0.1000 & 3.055 \\
		jpy & -0.027 & 0.0838 & -0.020 & 0.0659 & 3.094 \\
		gold & 0.009 & 0.140 & -0.034 & 0.1000 & 4.100 \\
		wti & 0.023 & 0.299 & -0.061 & 0.0002 & 3.953 \\
		grain & -0.041 & 0.206 & -0.046 & 0.0017 & 2.505 \\
		metal & -0.013 & 0.176 & -0.034 & 0.1000 & 2.215 \\
		\midrule
	\end{tabular}
\label{tab:merton_params}
\end{table}

The next step is to perform the calibration of the model correlation matrix by solving equation \eqref{eq:expected_observed} numerically, for each possible asset pair.
To do so, we used the function \texttt{uniroot} which is included in the \texttt{stats} package. \texttt{uniroot} implements a bisection method to find the zero of a univariate function $f(x)$ in a given interval $[a,b]$. The only thing we have to make sure of is that $f(a)f(b)< 0 $ otherwise the algorithm will return an error.
With highly positive or negative values of the observed correlation, there might not be a corresponding value in $[-1,1]$ for the model correlation that solves the equation. For this reason, whenever we encounter such an issue we set $\rho_{model}$ to be 1 or -1, depending on the sign of $\rho_{obs}$.

Finally, the different values of $\rho_{model}$ are stored in a matrix for which we perform J\"ackel's regularization.
The resulting correlation structure is shown in Table \ref{tab:merton_corr}, with the usual colour notation.


The entire procedure of calibration for the Merton model is fairly fast and only takes about 10 minutes when executed in our computer.



% Table generated by Excel2LaTeX from sheet 'Foglio2'
\begin{table}%[htbp]
  \centering
  \tiny
  \caption[Merton correlation matrix]{Resulting model correlation matrix for Merton. The values are in percentages and the colour goes from red for $\rho= 100\%$, to white for $\rho=0\%$ and to blue for $\rho=-100\%$.}
    \noindent\makebox[\textwidth]{\begin{tabular}{lrrrrrrrrrrrrrrrr}
          & \multicolumn{1}{c}{\begin{sideways}btc\end{sideways}} & \multicolumn{1}{c}{\begin{sideways}bric\end{sideways}} & \multicolumn{1}{c}{\begin{sideways}sp500\end{sideways}} & \multicolumn{1}{c}{\begin{sideways}eurostoxx\end{sideways}} & \multicolumn{1}{c}{\begin{sideways}nasdaq\end{sideways}} & \multicolumn{1}{c}{\begin{sideways}bond\_europe\end{sideways}} & \multicolumn{1}{c}{\begin{sideways}bond\_us\end{sideways}} & \multicolumn{1}{c}{\begin{sideways}bond\_eur\end{sideways}} & \multicolumn{1}{c}{\begin{sideways}eur\end{sideways}} & \multicolumn{1}{c}{\begin{sideways}gbp\end{sideways}} & \multicolumn{1}{c}{\begin{sideways}chf\end{sideways}} & \multicolumn{1}{c}{\begin{sideways}jpy\end{sideways}} & \multicolumn{1}{c}{\begin{sideways}gold\end{sideways}} & \multicolumn{1}{c}{\begin{sideways}wti\end{sideways}} & \multicolumn{1}{c}{\begin{sideways}grain\end{sideways}} & \multicolumn{1}{c}{\begin{sideways}metal\end{sideways}} \\
    btc   & \cellcolor[rgb]{ .973,  .412,  .42}100.0 & \cellcolor[rgb]{ .988,  .976,  .988}2.7 & \cellcolor[rgb]{ .988,  .945,  .957}7.7 & \cellcolor[rgb]{ .988,  .953,  .965}6.3 & \cellcolor[rgb]{ .988,  .953,  .965}6.4 & \cellcolor[rgb]{ .988,  .98,  .992}1.9 & \cellcolor[rgb]{ .973,  .976,  .992}-2.4 & \cellcolor[rgb]{ .988,  .976,  .988}2.7 & \cellcolor[rgb]{ .988,  .973,  .984}2.7 & \cellcolor[rgb]{ .988,  .984,  .996}1.2 & \cellcolor[rgb]{ .988,  .965,  .976}4.4 & \cellcolor[rgb]{ .976,  .976,  .992}-1.8 & \cellcolor[rgb]{ .984,  .984,  .996}-0.6 & \cellcolor[rgb]{ .988,  .984,  .996}1.1 & \cellcolor[rgb]{ .988,  .965,  .976}4.2 & \cellcolor[rgb]{ .988,  .965,  .976}4.2 \\
    bric  & \cellcolor[rgb]{ .988,  .976,  .988}2.7 & \cellcolor[rgb]{ .973,  .412,  .42}100.0 & \cellcolor[rgb]{ .976,  .502,  .51}84.9 & \cellcolor[rgb]{ .976,  .51,  .522}83.1 & \cellcolor[rgb]{ .976,  .455,  .463}92.8 & \cellcolor[rgb]{ .984,  .816,  .824}30.5 & \cellcolor[rgb]{ .827,  .875,  .941}-24.8 & \cellcolor[rgb]{ .984,  .812,  .82}31.1 & \cellcolor[rgb]{ .984,  .812,  .824}31.0 & \cellcolor[rgb]{ .984,  .761,  .773}39.8 & \cellcolor[rgb]{ .988,  .882,  .89}19.0 & \cellcolor[rgb]{ .773,  .835,  .922}-33.9 & \cellcolor[rgb]{ .988,  .847,  .859}24.7 & \cellcolor[rgb]{ .984,  .718,  .725}47.5 & \cellcolor[rgb]{ .988,  .855,  .867}23.4 & \cellcolor[rgb]{ .976,  .529,  .541}79.7 \\
    sp500 & \cellcolor[rgb]{ .988,  .945,  .957}7.7 & \cellcolor[rgb]{ .976,  .502,  .51}84.9 & \cellcolor[rgb]{ .973,  .412,  .42}100.0 & \cellcolor[rgb]{ .976,  .482,  .49}88.3 & \cellcolor[rgb]{ .976,  .447,  .455}94.1 & \cellcolor[rgb]{ .988,  .867,  .878}21.1 & \cellcolor[rgb]{ .588,  .706,  .859}-62.5 & \cellcolor[rgb]{ .988,  .851,  .863}24.3 & \cellcolor[rgb]{ .984,  .804,  .816}32.2 & \cellcolor[rgb]{ .984,  .761,  .769}40.1 & \cellcolor[rgb]{ .988,  .957,  .969}6.0 & \cellcolor[rgb]{ .639,  .741,  .875}-54.6 & \cellcolor[rgb]{ .988,  .988,  1}0.6 & \cellcolor[rgb]{ .98,  .62,  .627}64.5 & \cellcolor[rgb]{ .984,  .827,  .839}28.1 & \cellcolor[rgb]{ .98,  .573,  .584}72.2 \\
    eurostoxx & \cellcolor[rgb]{ .988,  .953,  .965}6.3 & \cellcolor[rgb]{ .976,  .51,  .522}83.1 & \cellcolor[rgb]{ .976,  .482,  .49}88.3 & \cellcolor[rgb]{ .973,  .412,  .42}100.0 & \cellcolor[rgb]{ .976,  .51,  .518}83.6 & \cellcolor[rgb]{ .98,  .647,  .659}59.2 & \cellcolor[rgb]{ .729,  .804,  .906}-40.6 & \cellcolor[rgb]{ .98,  .631,  .643}62.1 & \cellcolor[rgb]{ .98,  .6,  .608}67.9 & \cellcolor[rgb]{ .98,  .612,  .624}65.5 & \cellcolor[rgb]{ .984,  .729,  .741}45.1 & \cellcolor[rgb]{ .839,  .882,  .945}-23.2 & \cellcolor[rgb]{ .984,  .839,  .847}26.4 & \cellcolor[rgb]{ .984,  .718,  .729}47.1 & \cellcolor[rgb]{ .988,  .863,  .871}22.3 & \cellcolor[rgb]{ .976,  .553,  .565}75.6 \\
    nasdaq & \cellcolor[rgb]{ .988,  .953,  .965}6.4 & \cellcolor[rgb]{ .976,  .455,  .463}92.8 & \cellcolor[rgb]{ .976,  .447,  .455}94.1 & \cellcolor[rgb]{ .976,  .51,  .518}83.6 & \cellcolor[rgb]{ .973,  .412,  .42}100.0 & \cellcolor[rgb]{ .988,  .882,  .894}18.4 & \cellcolor[rgb]{ .643,  .745,  .878}-53.7 & \cellcolor[rgb]{ .988,  .871,  .882}20.9 & \cellcolor[rgb]{ .984,  .843,  .855}25.4 & \cellcolor[rgb]{ .984,  .796,  .808}33.7 & \cellcolor[rgb]{ .984,  .984,  .996}-0.2 & \cellcolor[rgb]{ .635,  .741,  .875}-55.0 & \cellcolor[rgb]{ .976,  .98,  .996}-1.4 & \cellcolor[rgb]{ .98,  .694,  .706}51.0 & \cellcolor[rgb]{ .988,  .847,  .859}24.5 & \cellcolor[rgb]{ .98,  .612,  .62}65.6 \\
    bond\_europe & \cellcolor[rgb]{ .988,  .98,  .992}1.9 & \cellcolor[rgb]{ .984,  .816,  .824}30.5 & \cellcolor[rgb]{ .988,  .867,  .878}21.1 & \cellcolor[rgb]{ .98,  .647,  .659}59.2 & \cellcolor[rgb]{ .988,  .882,  .894}18.4 & \cellcolor[rgb]{ .973,  .412,  .42}100.0 & \cellcolor[rgb]{ .988,  .875,  .886}20.1 & \cellcolor[rgb]{ .976,  .424,  .431}98.6 & \cellcolor[rgb]{ .976,  .435,  .443}96.2 & \cellcolor[rgb]{ .98,  .573,  .58}72.5 & \cellcolor[rgb]{ .976,  .459,  .467}92.1 & \cellcolor[rgb]{ .98,  .651,  .663}58.5 & \cellcolor[rgb]{ .98,  .588,  .596}69.9 & \cellcolor[rgb]{ .988,  .89,  .902}17.6 & \cellcolor[rgb]{ .988,  .914,  .925}13.3 & \cellcolor[rgb]{ .984,  .78,  .788}36.7 \\
    bond\_us & \cellcolor[rgb]{ .973,  .976,  .992}-2.4 & \cellcolor[rgb]{ .827,  .875,  .941}-24.8 & \cellcolor[rgb]{ .588,  .706,  .859}-62.5 & \cellcolor[rgb]{ .729,  .804,  .906}-40.6 & \cellcolor[rgb]{ .643,  .745,  .878}-53.7 & \cellcolor[rgb]{ .988,  .875,  .886}20.1 & \cellcolor[rgb]{ .973,  .412,  .42}100.0 & \cellcolor[rgb]{ .988,  .902,  .914}15.0 & \cellcolor[rgb]{ .98,  .984,  .996}-0.7 & \cellcolor[rgb]{ .945,  .957,  .984}-6.5 & \cellcolor[rgb]{ .984,  .824,  .835}28.6 & \cellcolor[rgb]{ .98,  .635,  .643}61.9 & \cellcolor[rgb]{ .984,  .773,  .78}38.0 & \cellcolor[rgb]{ .827,  .875,  .941}-25.2 & \cellcolor[rgb]{ .937,  .949,  .98}-8.0 & \cellcolor[rgb]{ .831,  .878,  .945}-24.4 \\
    bond\_eur & \cellcolor[rgb]{ .988,  .976,  .988}2.7 & \cellcolor[rgb]{ .984,  .812,  .82}31.1 & \cellcolor[rgb]{ .988,  .851,  .863}24.3 & \cellcolor[rgb]{ .98,  .631,  .643}62.1 & \cellcolor[rgb]{ .988,  .871,  .882}20.9 & \cellcolor[rgb]{ .976,  .424,  .431}98.6 & \cellcolor[rgb]{ .988,  .902,  .914}15.0 & \cellcolor[rgb]{ .973,  .412,  .42}100.0 & \cellcolor[rgb]{ .976,  .424,  .431}98.4 & \cellcolor[rgb]{ .98,  .62,  .627}64.5 & \cellcolor[rgb]{ .976,  .471,  .478}90.0 & \cellcolor[rgb]{ .98,  .678,  .686}54.2 & \cellcolor[rgb]{ .98,  .604,  .616}66.9 & \cellcolor[rgb]{ .988,  .89,  .902}17.5 & \cellcolor[rgb]{ .988,  .914,  .925}13.0 & \cellcolor[rgb]{ .984,  .765,  .773}39.3 \\
    eur   & \cellcolor[rgb]{ .988,  .973,  .984}2.7 & \cellcolor[rgb]{ .984,  .812,  .824}31.0 & \cellcolor[rgb]{ .984,  .804,  .816}32.2 & \cellcolor[rgb]{ .98,  .6,  .608}67.9 & \cellcolor[rgb]{ .984,  .843,  .855}25.4 & \cellcolor[rgb]{ .976,  .435,  .443}96.2 & \cellcolor[rgb]{ .98,  .984,  .996}-0.7 & \cellcolor[rgb]{ .976,  .424,  .431}98.4 & \cellcolor[rgb]{ .973,  .412,  .42}100.0 & \cellcolor[rgb]{ .98,  .596,  .608}68.1 & \cellcolor[rgb]{ .976,  .482,  .49}87.9 & \cellcolor[rgb]{ .984,  .722,  .733}46.3 & \cellcolor[rgb]{ .98,  .635,  .647}61.2 & \cellcolor[rgb]{ .988,  .871,  .882}20.9 & \cellcolor[rgb]{ .988,  .902,  .914}15.0 & \cellcolor[rgb]{ .984,  .745,  .757}42.3 \\
    gbp   & \cellcolor[rgb]{ .988,  .984,  .996}1.2 & \cellcolor[rgb]{ .984,  .761,  .773}39.8 & \cellcolor[rgb]{ .984,  .761,  .769}40.1 & \cellcolor[rgb]{ .98,  .612,  .624}65.5 & \cellcolor[rgb]{ .984,  .796,  .808}33.7 & \cellcolor[rgb]{ .98,  .573,  .58}72.5 & \cellcolor[rgb]{ .945,  .957,  .984}-6.5 & \cellcolor[rgb]{ .98,  .62,  .627}64.5 & \cellcolor[rgb]{ .98,  .596,  .608}68.1 & \cellcolor[rgb]{ .973,  .412,  .42}100.0 & \cellcolor[rgb]{ .98,  .592,  .6}69.1 & \cellcolor[rgb]{ .988,  .855,  .867}23.4 & \cellcolor[rgb]{ .984,  .737,  .749}43.6 & \cellcolor[rgb]{ .984,  .835,  .843}27.2 & \cellcolor[rgb]{ .988,  .906,  .918}14.6 & \cellcolor[rgb]{ .984,  .769,  .776}38.5 \\
    chf   & \cellcolor[rgb]{ .988,  .965,  .976}4.4 & \cellcolor[rgb]{ .988,  .882,  .89}19.0 & \cellcolor[rgb]{ .988,  .957,  .969}6.0 & \cellcolor[rgb]{ .984,  .729,  .741}45.1 & \cellcolor[rgb]{ .984,  .984,  .996}-0.2 & \cellcolor[rgb]{ .976,  .459,  .467}92.1 & \cellcolor[rgb]{ .984,  .824,  .835}28.6 & \cellcolor[rgb]{ .976,  .471,  .478}90.0 & \cellcolor[rgb]{ .976,  .482,  .49}87.9 & \cellcolor[rgb]{ .98,  .592,  .6}69.1 & \cellcolor[rgb]{ .973,  .412,  .42}100.0 & \cellcolor[rgb]{ .98,  .561,  .569}74.8 & \cellcolor[rgb]{ .976,  .49,  .498}86.7 & \cellcolor[rgb]{ .988,  .914,  .925}12.9 & \cellcolor[rgb]{ .988,  .906,  .918}14.4 & \cellcolor[rgb]{ .984,  .733,  .741}44.9 \\
    jpy   & \cellcolor[rgb]{ .976,  .976,  .992}-1.8 & \cellcolor[rgb]{ .773,  .835,  .922}-33.9 & \cellcolor[rgb]{ .639,  .741,  .875}-54.6 & \cellcolor[rgb]{ .839,  .882,  .945}-23.2 & \cellcolor[rgb]{ .635,  .741,  .875}-55.0 & \cellcolor[rgb]{ .98,  .651,  .663}58.5 & \cellcolor[rgb]{ .98,  .635,  .643}61.9 & \cellcolor[rgb]{ .98,  .678,  .686}54.2 & \cellcolor[rgb]{ .984,  .722,  .733}46.3 & \cellcolor[rgb]{ .988,  .855,  .867}23.4 & \cellcolor[rgb]{ .98,  .561,  .569}74.8 & \cellcolor[rgb]{ .973,  .412,  .42}100.0 & \cellcolor[rgb]{ .976,  .533,  .541}79.1 & \cellcolor[rgb]{ .91,  .933,  .973}-12.2 & \cellcolor[rgb]{ .988,  .973,  .984}3.0 & \cellcolor[rgb]{ .957,  .965,  .988}-4.6 \\
    gold  & \cellcolor[rgb]{ .984,  .984,  .996}-0.6 & \cellcolor[rgb]{ .988,  .847,  .859}24.7 & \cellcolor[rgb]{ .988,  .988,  1}0.6 & \cellcolor[rgb]{ .984,  .839,  .847}26.4 & \cellcolor[rgb]{ .976,  .98,  .996}-1.4 & \cellcolor[rgb]{ .98,  .588,  .596}69.9 & \cellcolor[rgb]{ .984,  .773,  .78}38.0 & \cellcolor[rgb]{ .98,  .604,  .616}66.9 & \cellcolor[rgb]{ .98,  .635,  .647}61.2 & \cellcolor[rgb]{ .984,  .737,  .749}43.6 & \cellcolor[rgb]{ .976,  .49,  .498}86.7 & \cellcolor[rgb]{ .976,  .533,  .541}79.1 & \cellcolor[rgb]{ .973,  .412,  .42}100.0 & \cellcolor[rgb]{ .988,  .851,  .863}24.2 & \cellcolor[rgb]{ .988,  .859,  .867}23.0 & \cellcolor[rgb]{ .98,  .667,  .675}56.3 \\
    wti   & \cellcolor[rgb]{ .988,  .984,  .996}1.1 & \cellcolor[rgb]{ .984,  .718,  .725}47.5 & \cellcolor[rgb]{ .98,  .62,  .627}64.5 & \cellcolor[rgb]{ .984,  .718,  .729}47.1 & \cellcolor[rgb]{ .98,  .694,  .706}51.0 & \cellcolor[rgb]{ .988,  .89,  .902}17.6 & \cellcolor[rgb]{ .827,  .875,  .941}-25.2 & \cellcolor[rgb]{ .988,  .89,  .902}17.5 & \cellcolor[rgb]{ .988,  .871,  .882}20.9 & \cellcolor[rgb]{ .984,  .835,  .843}27.2 & \cellcolor[rgb]{ .988,  .914,  .925}12.9 & \cellcolor[rgb]{ .91,  .933,  .973}-12.2 & \cellcolor[rgb]{ .988,  .851,  .863}24.2 & \cellcolor[rgb]{ .973,  .412,  .42}100.0 & \cellcolor[rgb]{ .988,  .871,  .882}20.6 & \cellcolor[rgb]{ .984,  .702,  .714}49.8 \\
    grain & \cellcolor[rgb]{ .988,  .965,  .976}4.2 & \cellcolor[rgb]{ .988,  .855,  .867}23.4 & \cellcolor[rgb]{ .984,  .827,  .839}28.1 & \cellcolor[rgb]{ .988,  .863,  .871}22.3 & \cellcolor[rgb]{ .988,  .847,  .859}24.5 & \cellcolor[rgb]{ .988,  .914,  .925}13.3 & \cellcolor[rgb]{ .937,  .949,  .98}-8.0 & \cellcolor[rgb]{ .988,  .914,  .925}13.0 & \cellcolor[rgb]{ .988,  .902,  .914}15.0 & \cellcolor[rgb]{ .988,  .906,  .918}14.6 & \cellcolor[rgb]{ .988,  .906,  .918}14.4 & \cellcolor[rgb]{ .988,  .973,  .984}3.0 & \cellcolor[rgb]{ .988,  .859,  .867}23.0 & \cellcolor[rgb]{ .988,  .871,  .882}20.6 & \cellcolor[rgb]{ .973,  .412,  .42}100.0 & \cellcolor[rgb]{ .984,  .827,  .839}28.1 \\
    metal & \cellcolor[rgb]{ .988,  .965,  .976}4.2 & \cellcolor[rgb]{ .976,  .529,  .541}79.7 & \cellcolor[rgb]{ .98,  .573,  .584}72.2 & \cellcolor[rgb]{ .976,  .553,  .565}75.6 & \cellcolor[rgb]{ .98,  .612,  .62}65.6 & \cellcolor[rgb]{ .984,  .78,  .788}36.7 & \cellcolor[rgb]{ .831,  .878,  .945}-24.4 & \cellcolor[rgb]{ .984,  .765,  .773}39.3 & \cellcolor[rgb]{ .984,  .745,  .757}42.3 & \cellcolor[rgb]{ .984,  .769,  .776}38.5 & \cellcolor[rgb]{ .984,  .733,  .741}44.9 & \cellcolor[rgb]{ .957,  .965,  .988}-4.6 & \cellcolor[rgb]{ .98,  .667,  .675}56.3 & \cellcolor[rgb]{ .984,  .702,  .714}49.8 & \cellcolor[rgb]{ .984,  .827,  .839}28.1 & \cellcolor[rgb]{ .973,  .412,  .42}100.0 \\
    \end{tabular}%
}
  \label{tab:merton_corr}
\end{table}%
\subsection{Heston Calibration Results}

The implementation of the calibration of Heston model for each asset is very similar to the one that we presented in the previous section for Merton. We use two optimizer, one global and one local.

The main difference in the maximization of the log-likelihood function is that now we have an extra constraint given by the Feller condition: the same will be true for the calibration of the parameters in Bates model.

To include this constraint in the global optimizer we used the package \texttt{DEoptimR} and the function \texttt{JDEoptim}: this performs the same differential evolution optimization but allows for the specification of extra constraints. The relative manual can be found in \citep{DEoptimR_manual}.

For the local optimization we still used the \texttt{nlminb} function: to include the constraint we added a penalty to the negative log-likelihood function that would set its value to a high negative number (e.g. $-10^8$).

The box constraints for the Heston parameters $\psi =  \left\{ \mu, \kappa, \theta, \sigma_V, \rho \right\}$ are given in the following table:

\bigskip

\begin{center}
	\begin{tabular}{lccccc}
		&$\mu$ & $\kappa$ & $\theta$ & $\sigma_V$ & $\rho$ \\
		\midrule
		lower bound & $\mu_{sample}-0.05$ & $10^{-3}$ & $10^{-3}$ &$10^{-5}$ &$10^{-4}$ or -1\\
		upper bound & $\mu_{sample}+0.05 $& 2 & 3 & 2& 1 or $-10^-4$ \\
		\midrule
	\end{tabular}
\end{center}

\bigskip

By running the calibration a few times we noticed that the resulting values of $\mu$ coming from the calibration procedure with the same bounds as in Merton were nonsensical both in magnitude and in sign. Thus, we resort to impose that the boundary for the drift is the expected return obtained from the sample with the possibility to move up or down of 0.05.

For the parameter $\rho$ that models the asymmetry of the return distribution we restrict the boundary to positive values when the skewness of the return distribution is positive and to negative values if the skewness is negative. 
The calibrated parameters for each asset are shown in Table \ref{tab:heston_params}.

The resulting density is graphically compared to the empirical distribution of the log-returns in Appendix \ref{app:hist}.

\bigskip

\begin{table}
	\small
	\centering
	\caption[Heston calibrated parameters]{Numerical results for the calibration of the parameters of the Heston model for each asset.}
	\label{tab:heston_params}
	\begin{tabular}{lrrrrr}
		
		& $\mu$ & $\kappa$ & $\theta$ & $\sigma_V$ & $\rho$ \\
		\midrule
		btc & 1.377 & 0.677 & 0.738 & 0.9998 & -0.0002 \\
		bric & -0.005 & 2.000 & 0.030 & 0.2588 & -1.0000 \\
		sp500 & 0.100 & 0.460 & 0.017 & 0.1267 & -0.0040 \\
		eurostoxx & 0.014 & 0.964 & 0.047 & 0.2977 & -0.3063 \\
		nasdaq & 0.153 & 1.286 & 0.025 & 0.2554 & -0.3667 \\
		bond\_europe & 0.028 & 0.621 & 0.007 & 0.0562 & 0.0031 \\
		bond\_us & 0.031 & 1.726 & 0.001 & 0.0378 & -0.9606 \\
		bond\_eur & 0.030 & 0.568 & 0.008 & 0.0638 & 0.0059 \\
		eur & -0.004 & 1.498 & 0.007 & 0.1077 & 0.0522 \\
		gbp & -0.023 & 1.291 & 0.006 & 0.0901 & -0.0175 \\
		chf & 0.011 & 0.001 & 0.008 & 0.0034 & 0.0001 \\
		jpy & -0.030 & 0.985 & 0.008 & 0.1045 & -0.9874 \\
		gold & 0.008 & 0.364 & 0.023 & 0.1234 & -1.0000 \\
		wti & -0.009 & 0.293 & 0.103 & 0.2460 & 0.0001 \\
		grain & -0.069 & 2.000 & 0.046 & 0.3348 & 0.7014 \\
		metal & -0.032 & 0.001 & 0.035 & 0.0079 & 0.0001 \\
		\midrule
	\end{tabular}
\end{table}


Next we perform the calibration of the  $\rho_{model}$ for each pair of assets, by using the same algorithm coded in \texttt{uniroot} that we presented  in the  previous section. Again, in case that the observed sample correlation is too high in absolute value, we set $\rho_{model}$ equal to 1 or -1.

Lastly, we perform the usual regularization to obtain a valid model correlation matrix.
The final result is shown in Table \ref{tab:heston_corr}.

\bigskip
The increase in the complexity in terms of the formulation of the model generates a great increase in the computation time for the calibration. Each time we have to compute the (negative) log-likelihood function we need to invert the characteristic function: even when taking advantage of the FFT algorithm there still is a great increase with respect to the calibration time in Merton's case. The total time amounts to about three hour, with a single asset calibration taking up to fifteen minutes. \textcolor{red}{aggiungere caratteristiche computer}

\begin{table}
	\tiny
  \centering
  \caption[Heston correlation matrix]{Resulting correlation matrix for Heston. The values are percentages and the colour scheme is the usual.}
    \noindent\makebox[\textwidth]{\begin{tabular}{lrrrrrrrrrrrrrrrr}
          & \multicolumn{1}{c}{\begin{sideways}btc\end{sideways}} & \multicolumn{1}{c}{\begin{sideways}bric\end{sideways}} & \multicolumn{1}{c}{\begin{sideways}sp500\end{sideways}} & \multicolumn{1}{c}{\begin{sideways}eurostoxx\end{sideways}} & \multicolumn{1}{c}{\begin{sideways}nasdaq\end{sideways}} & \multicolumn{1}{c}{\begin{sideways}bond\_europe\end{sideways}} & \multicolumn{1}{c}{\begin{sideways}bond\_us\end{sideways}} & \multicolumn{1}{c}{\begin{sideways}bond\_eur\end{sideways}} & \multicolumn{1}{c}{\begin{sideways}eur\end{sideways}} & \multicolumn{1}{c}{\begin{sideways}gbp\end{sideways}} & \multicolumn{1}{c}{\begin{sideways}chf\end{sideways}} & \multicolumn{1}{c}{\begin{sideways}jpy\end{sideways}} & \multicolumn{1}{c}{\begin{sideways}gold\end{sideways}} & \multicolumn{1}{c}{\begin{sideways}wti\end{sideways}} & \multicolumn{1}{c}{\begin{sideways}grain\end{sideways}} & \multicolumn{1}{c}{\begin{sideways}metal\end{sideways}} \\
    btc   & \cellcolor[rgb]{ .973,  .412,  .42}100.0 & \cellcolor[rgb]{ .988,  .98,  .992}1.8 & \cellcolor[rgb]{ .988,  .953,  .965}6.3 & \cellcolor[rgb]{ .988,  .957,  .969}5.9 & \cellcolor[rgb]{ .988,  .961,  .969}5.4 & \cellcolor[rgb]{ .988,  .98,  .992}1.9 & \cellcolor[rgb]{ .973,  .976,  .992}-2.2 & \cellcolor[rgb]{ .988,  .976,  .988}2.6 & \cellcolor[rgb]{ .988,  .973,  .984}2.8 & \cellcolor[rgb]{ .988,  .984,  .996}0.9 & \cellcolor[rgb]{ .988,  .973,  .984}3.2 & \cellcolor[rgb]{ .976,  .976,  .992}-1.8 & \cellcolor[rgb]{ .984,  .984,  .996}-0.5 & \cellcolor[rgb]{ .988,  .984,  .996}0.9 & \cellcolor[rgb]{ .988,  .965,  .976}4.5 & \cellcolor[rgb]{ .988,  .973,  .984}3.3 \\
    bric  & \cellcolor[rgb]{ .988,  .98,  .992}1.8 & \cellcolor[rgb]{ .973,  .412,  .42}100.0 & \cellcolor[rgb]{ .98,  .647,  .659}59.5 & \cellcolor[rgb]{ .98,  .604,  .612}67.3 & \cellcolor[rgb]{ .98,  .655,  .667}58.0 & \cellcolor[rgb]{ .988,  .863,  .875}22.0 & \cellcolor[rgb]{ .878,  .91,  .961}-17.0 & \cellcolor[rgb]{ .988,  .863,  .875}22.0 & \cellcolor[rgb]{ .988,  .863,  .875}22.0 & \cellcolor[rgb]{ .984,  .835,  .847}26.8 & \cellcolor[rgb]{ .988,  .937,  .949}9.2 & \cellcolor[rgb]{ .863,  .898,  .953}-19.7 & \cellcolor[rgb]{ .988,  .902,  .914}15.5 & \cellcolor[rgb]{ .984,  .796,  .804}33.8 & \cellcolor[rgb]{ .988,  .89,  .902}17.1 & \cellcolor[rgb]{ .984,  .725,  .733}46.3 \\
    sp500 & \cellcolor[rgb]{ .988,  .953,  .965}6.3 & \cellcolor[rgb]{ .98,  .647,  .659}59.5 & \cellcolor[rgb]{ .973,  .412,  .42}100.0 & \cellcolor[rgb]{ .976,  .537,  .549}78.3 & \cellcolor[rgb]{ .976,  .42,  .427}99.0 & \cellcolor[rgb]{ .988,  .898,  .91}15.8 & \cellcolor[rgb]{ .729,  .804,  .906}-40.5 & \cellcolor[rgb]{ .988,  .886,  .898}17.8 & \cellcolor[rgb]{ .988,  .867,  .878}21.5 & \cellcolor[rgb]{ .984,  .843,  .855}25.3 & \cellcolor[rgb]{ .988,  .988,  1}0.4 & \cellcolor[rgb]{ .808,  .863,  .937}-27.9 & \cellcolor[rgb]{ .98,  .98,  .996}-0.9 & \cellcolor[rgb]{ .984,  .749,  .761}41.8 & \cellcolor[rgb]{ .988,  .886,  .894}18.3 & \cellcolor[rgb]{ .984,  .761,  .773}39.8 \\
    eurostoxx & \cellcolor[rgb]{ .988,  .957,  .969}5.9 & \cellcolor[rgb]{ .98,  .604,  .612}67.3 & \cellcolor[rgb]{ .976,  .537,  .549}78.3 & \cellcolor[rgb]{ .973,  .412,  .42}100.0 & \cellcolor[rgb]{ .98,  .58,  .588}71.2 & \cellcolor[rgb]{ .984,  .71,  .722}48.5 & \cellcolor[rgb]{ .78,  .843,  .925}-32.2 & \cellcolor[rgb]{ .98,  .698,  .706}51.0 & \cellcolor[rgb]{ .98,  .675,  .682}54.9 & \cellcolor[rgb]{ .984,  .714,  .722}48.3 & \cellcolor[rgb]{ .988,  .847,  .859}24.9 & \cellcolor[rgb]{ .855,  .894,  .953}-20.4 & \cellcolor[rgb]{ .988,  .925,  .937}11.3 & \cellcolor[rgb]{ .984,  .773,  .784}37.6 & \cellcolor[rgb]{ .988,  .89,  .902}17.5 & \cellcolor[rgb]{ .98,  .694,  .702}51.5 \\
    nasdaq & \cellcolor[rgb]{ .988,  .961,  .969}5.4 & \cellcolor[rgb]{ .98,  .655,  .667}58.0 & \cellcolor[rgb]{ .976,  .42,  .427}99.0 & \cellcolor[rgb]{ .98,  .58,  .588}71.2 & \cellcolor[rgb]{ .973,  .412,  .42}100.0 & \cellcolor[rgb]{ .988,  .918,  .925}12.9 & \cellcolor[rgb]{ .749,  .82,  .914}-37.0 & \cellcolor[rgb]{ .988,  .906,  .918}14.4 & \cellcolor[rgb]{ .988,  .89,  .898}17.6 & \cellcolor[rgb]{ .988,  .863,  .875}22.2 & \cellcolor[rgb]{ .973,  .976,  .992}-2.0 & \cellcolor[rgb]{ .82,  .867,  .937}-26.5 & \cellcolor[rgb]{ .976,  .98,  .996}-1.5 & \cellcolor[rgb]{ .984,  .792,  .8}34.7 & \cellcolor[rgb]{ .988,  .894,  .906}16.9 & \cellcolor[rgb]{ .984,  .776,  .788}36.9 \\
    bond\_europe & \cellcolor[rgb]{ .988,  .98,  .992}1.9 & \cellcolor[rgb]{ .988,  .863,  .875}22.0 & \cellcolor[rgb]{ .988,  .898,  .91}15.8 & \cellcolor[rgb]{ .984,  .71,  .722}48.5 & \cellcolor[rgb]{ .988,  .918,  .925}12.9 & \cellcolor[rgb]{ .973,  .412,  .42}100.0 & \cellcolor[rgb]{ .988,  .871,  .882}20.7 & \cellcolor[rgb]{ .976,  .42,  .427}98.9 & \cellcolor[rgb]{ .976,  .435,  .443}96.1 & \cellcolor[rgb]{ .98,  .604,  .612}67.0 & \cellcolor[rgb]{ .98,  .624,  .631}63.7 & \cellcolor[rgb]{ .984,  .741,  .749}43.4 & \cellcolor[rgb]{ .984,  .702,  .714}49.8 & \cellcolor[rgb]{ .988,  .902,  .91}15.6 & \cellcolor[rgb]{ .988,  .918,  .929}12.4 & \cellcolor[rgb]{ .984,  .831,  .839}27.9 \\
    bond\_us & \cellcolor[rgb]{ .973,  .976,  .992}-2.2 & \cellcolor[rgb]{ .878,  .91,  .961}-17.0 & \cellcolor[rgb]{ .729,  .804,  .906}-40.5 & \cellcolor[rgb]{ .78,  .843,  .925}-32.2 & \cellcolor[rgb]{ .749,  .82,  .914}-37.0 & \cellcolor[rgb]{ .988,  .871,  .882}20.7 & \cellcolor[rgb]{ .973,  .412,  .42}100.0 & \cellcolor[rgb]{ .988,  .902,  .914}15.4 & \cellcolor[rgb]{ .984,  .984,  .996}-0.6 & \cellcolor[rgb]{ .949,  .961,  .984}-5.7 & \cellcolor[rgb]{ .988,  .902,  .914}15.1 & \cellcolor[rgb]{ .984,  .741,  .753}43.0 & \cellcolor[rgb]{ .988,  .851,  .859}24.4 & \cellcolor[rgb]{ .843,  .886,  .949}-22.6 & \cellcolor[rgb]{ .941,  .953,  .98}-7.3 & \cellcolor[rgb]{ .875,  .906,  .957}-17.6 \\
    bond\_eur & \cellcolor[rgb]{ .988,  .976,  .988}2.6 & \cellcolor[rgb]{ .988,  .863,  .875}22.0 & \cellcolor[rgb]{ .988,  .886,  .898}17.8 & \cellcolor[rgb]{ .98,  .698,  .706}51.0 & \cellcolor[rgb]{ .988,  .906,  .918}14.4 & \cellcolor[rgb]{ .976,  .42,  .427}98.9 & \cellcolor[rgb]{ .988,  .902,  .914}15.4 & \cellcolor[rgb]{ .973,  .412,  .42}100.0 & \cellcolor[rgb]{ .976,  .427,  .435}97.9 & \cellcolor[rgb]{ .98,  .659,  .667}57.7 & \cellcolor[rgb]{ .98,  .635,  .643}61.6 & \cellcolor[rgb]{ .984,  .757,  .769}40.2 & \cellcolor[rgb]{ .984,  .722,  .729}46.9 & \cellcolor[rgb]{ .988,  .902,  .914}15.3 & \cellcolor[rgb]{ .988,  .922,  .933}11.9 & \cellcolor[rgb]{ .984,  .824,  .835}28.8 \\
    eur   & \cellcolor[rgb]{ .988,  .973,  .984}2.8 & \cellcolor[rgb]{ .988,  .863,  .875}22.0 & \cellcolor[rgb]{ .988,  .867,  .878}21.5 & \cellcolor[rgb]{ .98,  .675,  .682}54.9 & \cellcolor[rgb]{ .988,  .89,  .898}17.6 & \cellcolor[rgb]{ .976,  .435,  .443}96.1 & \cellcolor[rgb]{ .984,  .984,  .996}-0.6 & \cellcolor[rgb]{ .976,  .427,  .435}97.9 & \cellcolor[rgb]{ .973,  .412,  .42}100.0 & \cellcolor[rgb]{ .98,  .639,  .647}60.8 & \cellcolor[rgb]{ .98,  .627,  .639}62.6 & \cellcolor[rgb]{ .984,  .792,  .804}34.2 & \cellcolor[rgb]{ .984,  .749,  .761}41.7 & \cellcolor[rgb]{ .988,  .878,  .89}19.1 & \cellcolor[rgb]{ .988,  .91,  .922}14.1 & \cellcolor[rgb]{ .984,  .808,  .82}31.5 \\
    gbp   & \cellcolor[rgb]{ .988,  .984,  .996}0.9 & \cellcolor[rgb]{ .984,  .835,  .847}26.8 & \cellcolor[rgb]{ .984,  .843,  .855}25.3 & \cellcolor[rgb]{ .984,  .714,  .722}48.3 & \cellcolor[rgb]{ .988,  .863,  .875}22.2 & \cellcolor[rgb]{ .98,  .604,  .612}67.0 & \cellcolor[rgb]{ .949,  .961,  .984}-5.7 & \cellcolor[rgb]{ .98,  .659,  .667}57.7 & \cellcolor[rgb]{ .98,  .639,  .647}60.8 & \cellcolor[rgb]{ .973,  .412,  .42}100.0 & \cellcolor[rgb]{ .984,  .765,  .773}39.3 & \cellcolor[rgb]{ .988,  .898,  .906}16.3 & \cellcolor[rgb]{ .984,  .824,  .835}28.7 & \cellcolor[rgb]{ .988,  .851,  .863}23.9 & \cellcolor[rgb]{ .988,  .918,  .929}12.7 & \cellcolor[rgb]{ .984,  .831,  .843}27.4 \\
    chf   & \cellcolor[rgb]{ .988,  .973,  .984}3.2 & \cellcolor[rgb]{ .988,  .937,  .949}9.2 & \cellcolor[rgb]{ .988,  .988,  1}0.4 & \cellcolor[rgb]{ .988,  .847,  .859}24.9 & \cellcolor[rgb]{ .973,  .976,  .992}-2.0 & \cellcolor[rgb]{ .98,  .624,  .631}63.7 & \cellcolor[rgb]{ .988,  .902,  .914}15.1 & \cellcolor[rgb]{ .98,  .635,  .643}61.6 & \cellcolor[rgb]{ .98,  .627,  .639}62.6 & \cellcolor[rgb]{ .984,  .765,  .773}39.3 & \cellcolor[rgb]{ .973,  .412,  .42}100.0 & \cellcolor[rgb]{ .984,  .757,  .769}40.3 & \cellcolor[rgb]{ .984,  .741,  .749}43.5 & \cellcolor[rgb]{ .988,  .949,  .961}6.9 & \cellcolor[rgb]{ .988,  .945,  .957}7.8 & \cellcolor[rgb]{ .988,  .867,  .875}21.7 \\
    jpy   & \cellcolor[rgb]{ .976,  .976,  .992}-1.8 & \cellcolor[rgb]{ .863,  .898,  .953}-19.7 & \cellcolor[rgb]{ .808,  .863,  .937}-27.9 & \cellcolor[rgb]{ .855,  .894,  .953}-20.4 & \cellcolor[rgb]{ .82,  .867,  .937}-26.5 & \cellcolor[rgb]{ .984,  .741,  .749}43.4 & \cellcolor[rgb]{ .984,  .741,  .753}43.0 & \cellcolor[rgb]{ .984,  .757,  .769}40.2 & \cellcolor[rgb]{ .984,  .792,  .804}34.2 & \cellcolor[rgb]{ .988,  .898,  .906}16.3 & \cellcolor[rgb]{ .984,  .757,  .769}40.3 & \cellcolor[rgb]{ .973,  .412,  .42}100.0 & \cellcolor[rgb]{ .984,  .71,  .722}48.4 & \cellcolor[rgb]{ .937,  .953,  .98}-7.4 & \cellcolor[rgb]{ .988,  .976,  .988}2.3 & \cellcolor[rgb]{ .965,  .973,  .992}-3.3 \\
    gold  & \cellcolor[rgb]{ .984,  .984,  .996}-0.5 & \cellcolor[rgb]{ .988,  .902,  .914}15.5 & \cellcolor[rgb]{ .98,  .98,  .996}-0.9 & \cellcolor[rgb]{ .988,  .925,  .937}11.3 & \cellcolor[rgb]{ .976,  .98,  .996}-1.5 & \cellcolor[rgb]{ .984,  .702,  .714}49.8 & \cellcolor[rgb]{ .988,  .851,  .859}24.4 & \cellcolor[rgb]{ .984,  .722,  .729}46.9 & \cellcolor[rgb]{ .984,  .749,  .761}41.7 & \cellcolor[rgb]{ .984,  .824,  .835}28.7 & \cellcolor[rgb]{ .984,  .741,  .749}43.5 & \cellcolor[rgb]{ .984,  .71,  .722}48.4 & \cellcolor[rgb]{ .973,  .412,  .42}100.0 & \cellcolor[rgb]{ .988,  .89,  .902}17.1 & \cellcolor[rgb]{ .988,  .898,  .91}15.9 & \cellcolor[rgb]{ .984,  .784,  .796}35.7 \\
    wti   & \cellcolor[rgb]{ .988,  .984,  .996}0.9 & \cellcolor[rgb]{ .984,  .796,  .804}33.8 & \cellcolor[rgb]{ .984,  .749,  .761}41.8 & \cellcolor[rgb]{ .984,  .773,  .784}37.6 & \cellcolor[rgb]{ .984,  .792,  .8}34.7 & \cellcolor[rgb]{ .988,  .902,  .91}15.6 & \cellcolor[rgb]{ .843,  .886,  .949}-22.6 & \cellcolor[rgb]{ .988,  .902,  .914}15.3 & \cellcolor[rgb]{ .988,  .878,  .89}19.1 & \cellcolor[rgb]{ .988,  .851,  .863}23.9 & \cellcolor[rgb]{ .988,  .949,  .961}6.9 & \cellcolor[rgb]{ .937,  .953,  .98}-7.4 & \cellcolor[rgb]{ .988,  .89,  .902}17.1 & \cellcolor[rgb]{ .973,  .412,  .42}100.0 & \cellcolor[rgb]{ .988,  .878,  .89}19.1 & \cellcolor[rgb]{ .984,  .773,  .784}37.4 \\
    grain & \cellcolor[rgb]{ .988,  .965,  .976}4.5 & \cellcolor[rgb]{ .988,  .89,  .902}17.1 & \cellcolor[rgb]{ .988,  .886,  .894}18.3 & \cellcolor[rgb]{ .988,  .89,  .902}17.5 & \cellcolor[rgb]{ .988,  .894,  .906}16.9 & \cellcolor[rgb]{ .988,  .918,  .929}12.4 & \cellcolor[rgb]{ .941,  .953,  .98}-7.3 & \cellcolor[rgb]{ .988,  .922,  .933}11.9 & \cellcolor[rgb]{ .988,  .91,  .922}14.1 & \cellcolor[rgb]{ .988,  .918,  .929}12.7 & \cellcolor[rgb]{ .988,  .945,  .957}7.8 & \cellcolor[rgb]{ .988,  .976,  .988}2.3 & \cellcolor[rgb]{ .988,  .898,  .91}15.9 & \cellcolor[rgb]{ .988,  .878,  .89}19.1 & \cellcolor[rgb]{ .973,  .412,  .42}100.0 & \cellcolor[rgb]{ .988,  .867,  .878}21.4 \\
    metal & \cellcolor[rgb]{ .988,  .973,  .984}3.3 & \cellcolor[rgb]{ .984,  .725,  .733}46.3 & \cellcolor[rgb]{ .984,  .761,  .773}39.8 & \cellcolor[rgb]{ .98,  .694,  .702}51.5 & \cellcolor[rgb]{ .984,  .776,  .788}36.9 & \cellcolor[rgb]{ .984,  .831,  .839}27.9 & \cellcolor[rgb]{ .875,  .906,  .957}-17.6 & \cellcolor[rgb]{ .984,  .824,  .835}28.8 & \cellcolor[rgb]{ .984,  .808,  .82}31.5 & \cellcolor[rgb]{ .984,  .831,  .843}27.4 & \cellcolor[rgb]{ .988,  .867,  .875}21.7 & \cellcolor[rgb]{ .965,  .973,  .992}-3.3 & \cellcolor[rgb]{ .984,  .784,  .796}35.7 & \cellcolor[rgb]{ .984,  .773,  .784}37.4 & \cellcolor[rgb]{ .988,  .867,  .878}21.4 & \cellcolor[rgb]{ .973,  .412,  .42}100.0 \\
    \end{tabular}%
}
  \label{tab:heston_corr}%
\end{table}%



\subsection{Bates Calibration Results}
The procedure to calibrate the parameters $\psi =  \left\{ \mu, \kappa, \theta, \sigma_V, \rho, \mu_J, \sigma_J, \lambda \right\}$ for Bates model we proceed exactly in the same way as we did for Heston. We first perform a global optimization using \texttt{JDEoptim} and then a local one through \texttt{nlminb} adding a penalty to the neg-log-likelihood function. 

The boundaries for each parameter are the combination of the intervals that we use for Merton and Heston and are reported in the following table:

\bigskip 

\begin{center}
	\noindent\makebox[\textwidth]{
		\begin{tabular}{lcccccccc}
		&$\mu$ &$\kappa$&$\theta$& $\sigma_V$ & $\rho$& $\mu_J$ & $\sigma_J$ & $\lambda$ \\
		\midrule
		lower bound & $\mu_{sample}-0.05$ & $10^{-3}$ & $10^{-3}$ &$10^{-5}$ &$10^{-4}$ or -1 & $q_{\alpha_{low}}(r_{sample})$ &$10^{-4}$ &$10^{-5}$\\
		upper bound & $\mu_{sample}-0.05$ & 2 & 3 & 2 &1 or $-10^{-4}$ & $q_{\alpha_{up}}(r_{sample})$ & 0.1& 10\\
		\midrule
	\end{tabular}
	}
\end{center}

\bigskip

We also introduce conditional bounds on the skewness for the $\mu_J$ variable: we set $\alpha_{low} = 0.1\%$ and $\alpha_{up} = 0.5\%$ if the skewness is negative and $\alpha_{low} = 99.5\%$ and $\alpha_{up} = 99.9\%$.
The parameters for each single asset in the Bates case are shown in Table \ref{tab:bates_params}.

The increased number of parameters causes an increase complexity in the shape of the log-likelihood function and hence it is harder for the optimizers to find the exact global optimum.



\begin{table}
	\small
	\centering
	\caption[Bates calibrated parameters]{Numerical results for the calibration of the parameters of the Bates model for each asset.}
	\label{tab:bates_params}
	\noindent\makebox[\textwidth]{
	\begin{tabular}{lrrrrrrrr}
		& $\mu$ & $\kappa$ & $\theta$ & $\sigma_V$ & $\rho$ & $\mu_J$ & $\sigma_J$ & $\lambda$ \\
		\midrule
		btc & 1.380 & 0.879 & 0.6703 & 1.0855 & -0.0036 & -0.2388 & 0.09973 & 0.928 \\
		bric & -0.005 & 2.000 & 0.0290 & 0.2443 & -0.9381 & -0.0600 & 0.00001 & 0.327 \\
		sp500 & 0.100 & 1.636 & 0.0168 & 0.2344 & -0.0019 & -0.0340 & 0.00020 & 1.342 \\
		eurostoxx & 0.002 & 1.303 & 0.0456 & 0.3311 & -0.1511 & -0.0506 & 0.00072 & 0.812 \\
		nasdaq & 0.135 & 1.289 & 0.0242 & 0.2498 & -0.0001 & -0.0368 & 0.00012 & 1.478 \\
		bond\_europe & 0.031 & 2.000 & 0.0066 & 0.1005 & 0.0001 & 0.0197 & 0.00001 & 0.196 \\
		bond\_us & 0.023 & 1.951 & 0.0010 & 0.0424 & -0.9963 & -0.0134 & 0.01021 & 0.001 \\
		bond\_eur & 0.028 & 1.389 & 0.0076 & 0.0960 & 0.0018 & 0.0186 & 0.01057 & 0.006 \\
		eur & -0.004 & 0.318 & 0.0071 & 0.0477 & 0.1526 & 0.0250 & 0.00001 & 0.163 \\
		gbp & -0.021 & 2.000 & 0.0060 & 0.0938 & -1.0000 & -0.0149 & 0.00119 & 0.805 \\
		chf & 0.005 & 0.932 & 0.0079 & 0.0925 & 0.7125 & 0.0215 & 0.00497 & 0.731 \\
		jpy & -0.034 & 2.000 & 0.0075 & 0.1360 & -0.0001 & -0.0250 & 0.00001 & 1.026 \\
		gold & 0.008 & 2.000 & 0.0220 & 0.2668 & -0.0467 & -0.0343 & 0.00089 & 1.166 \\
		wti & -0.009 & 0.278 & 0.1035 & 0.2400 & 0.0001 & 0.0654 & 0.00347 & 0.000 \\
		grain & -0.069 & 0.235 & 0.0443 & 0.1080 & 1.0000 & 0.0628 & 0.00001 & 0.571 \\
		metal & -0.032 & 0.001 & 0.0349 & 0.0079 & 0.0001 & 0.0369 & 0.00044 & 0.000 \\
		\midrule
	\end{tabular}
}
\end{table}

As for the calibration of the model correlation, we maintain the same approach of using \texttt{uniroot} to solve equation \eqref{eq:expected_observed} and setting $\rho_{model}$ to 1 or -1 in case the solver cannot come up with a solution.
The regularization using J\"ackel algorithm represents as usual the final step to obtain the full model correlation. The resulting matrix is reported in Table \ref{tab:bates_corr}.


\begin{table}  
	\tiny
	\centering
  
	\caption[Bates correlation matrix]{Resulting correlation matrix for Bates. Values are percentages and the colour scheme is the same used for the other correlation matrices.}
	\label{tab:bates_corr}
    
	\noindent\makebox[\textwidth]{
	\begin{tabular}{lrrrrrrrrrrrrrrrr}
 
		         & \multicolumn{1}{c}{\begin{sideways}btc\end{sideways}} & \multicolumn{1}{c}{\begin{sideways}bric\end{sideways}} & \multicolumn{1}{c}{\begin{sideways}sp500\end{sideways}} & \multicolumn{1}{c}{\begin{sideways}eurostoxx\end{sideways}} & \multicolumn{1}{c}{\begin{sideways}nasdaq\end{sideways}} & \multicolumn{1}{c}{\begin{sideways}bond\_europe\end{sideways}} & \multicolumn{1}{c}{\begin{sideways}bond\_us\end{sideways}} & \multicolumn{1}{c}{\begin{sideways}bond\_eur\end{sideways}} & \multicolumn{1}{c}{\begin{sideways}eur\end{sideways}} & \multicolumn{1}{c}{\begin{sideways}gbp\end{sideways}} & \multicolumn{1}{c}{\begin{sideways}chf\end{sideways}} & \multicolumn{1}{c}{\begin{sideways}jpy\end{sideways}} & \multicolumn{1}{c}{\begin{sideways}gold\end{sideways}} & \multicolumn{1}{c}{\begin{sideways}wti\end{sideways}} & \multicolumn{1}{c}{\begin{sideways}grain\end{sideways}} & \multicolumn{1}{c}{\begin{sideways}metal\end{sideways}} \\
    btc   & \cellcolor[rgb]{ .973,  .412,  .42}100.0 & \cellcolor[rgb]{ .988,  .98,  .992}1.5 & \cellcolor[rgb]{ .988,  .961,  .973}5.2 & \cellcolor[rgb]{ .988,  .965,  .976}4.7 & \cellcolor[rgb]{ .988,  .965,  .976}4.2 & \cellcolor[rgb]{ .988,  .98,  .992}1.4 & \cellcolor[rgb]{ .973,  .976,  .992}-2.3 & \cellcolor[rgb]{ .988,  .976,  .988}2.2 & \cellcolor[rgb]{ .988,  .973,  .984}3.1 & \cellcolor[rgb]{ .988,  .988,  1}0.5 & \cellcolor[rgb]{ .988,  .976,  .988}2.5 & \cellcolor[rgb]{ .98,  .98,  .996}-1.2 & \cellcolor[rgb]{ .984,  .984,  .996}-0.1 & \cellcolor[rgb]{ .988,  .984,  .996}1.0 & \cellcolor[rgb]{ .988,  .973,  .984}3.3 & \cellcolor[rgb]{ .988,  .973,  .984}2.8 \\
    bric  & \cellcolor[rgb]{ .988,  .98,  .992}1.5 & \cellcolor[rgb]{ .973,  .412,  .42}100.0 & \cellcolor[rgb]{ .98,  .69,  .702}51.8 & \cellcolor[rgb]{ .98,  .635,  .643}61.8 & \cellcolor[rgb]{ .98,  .698,  .706}50.7 & \cellcolor[rgb]{ .988,  .871,  .882}20.6 & \cellcolor[rgb]{ .882,  .914,  .961}-16.3 & \cellcolor[rgb]{ .988,  .871,  .878}21.1 & \cellcolor[rgb]{ .988,  .867,  .875}21.6 & \cellcolor[rgb]{ .984,  .839,  .851}25.9 & \cellcolor[rgb]{ .988,  .941,  .953}8.2 & \cellcolor[rgb]{ .875,  .91,  .961}-17.3 & \cellcolor[rgb]{ .988,  .91,  .922}13.8 & \cellcolor[rgb]{ .984,  .804,  .816}32.3 & \cellcolor[rgb]{ .988,  .894,  .906}16.4 & \cellcolor[rgb]{ .984,  .733,  .741}44.8 \\
    sp500 & \cellcolor[rgb]{ .988,  .961,  .973}5.2 & \cellcolor[rgb]{ .98,  .69,  .702}51.8 & \cellcolor[rgb]{ .973,  .412,  .42}100.0 & \cellcolor[rgb]{ .98,  .604,  .612}67.2 & \cellcolor[rgb]{ .976,  .416,  .424}99.6 & \cellcolor[rgb]{ .988,  .91,  .922}13.8 & \cellcolor[rgb]{ .757,  .827,  .918}-35.9 & \cellcolor[rgb]{ .988,  .902,  .914}15.4 & \cellcolor[rgb]{ .988,  .878,  .89}19.5 & \cellcolor[rgb]{ .988,  .863,  .875}22.0 & \cellcolor[rgb]{ .988,  .988,  1}0.2 & \cellcolor[rgb]{ .839,  .882,  .945}-23.0 & \cellcolor[rgb]{ .984,  .984,  .996}-0.5 & \cellcolor[rgb]{ .984,  .776,  .788}36.9 & \cellcolor[rgb]{ .988,  .894,  .906}16.4 & \cellcolor[rgb]{ .984,  .784,  .792}35.8 \\
    eurostoxx & \cellcolor[rgb]{ .988,  .965,  .976}4.7 & \cellcolor[rgb]{ .98,  .635,  .643}61.8 & \cellcolor[rgb]{ .98,  .604,  .612}67.2 & \cellcolor[rgb]{ .973,  .412,  .42}100.0 & \cellcolor[rgb]{ .98,  .631,  .643}61.9 & \cellcolor[rgb]{ .984,  .733,  .745}44.4 & \cellcolor[rgb]{ .796,  .855,  .933}-29.7 & \cellcolor[rgb]{ .984,  .718,  .725}47.4 & \cellcolor[rgb]{ .98,  .686,  .698}52.4 & \cellcolor[rgb]{ .984,  .729,  .741}45.3 & \cellcolor[rgb]{ .988,  .863,  .871}22.4 & \cellcolor[rgb]{ .875,  .91,  .961}-17.5 & \cellcolor[rgb]{ .988,  .933,  .945}9.8 & \cellcolor[rgb]{ .984,  .788,  .8}34.9 & \cellcolor[rgb]{ .988,  .894,  .906}16.9 & \cellcolor[rgb]{ .984,  .706,  .718}49.1 \\
    nasdaq & \cellcolor[rgb]{ .988,  .965,  .976}4.2 & \cellcolor[rgb]{ .98,  .698,  .706}50.7 & \cellcolor[rgb]{ .976,  .416,  .424}99.6 & \cellcolor[rgb]{ .98,  .631,  .643}61.9 & \cellcolor[rgb]{ .973,  .412,  .42}100.0 & \cellcolor[rgb]{ .988,  .925,  .937}11.4 & \cellcolor[rgb]{ .769,  .835,  .922}-34.1 & \cellcolor[rgb]{ .988,  .918,  .929}12.7 & \cellcolor[rgb]{ .988,  .894,  .906}16.5 & \cellcolor[rgb]{ .988,  .875,  .886}20.0 & \cellcolor[rgb]{ .973,  .976,  .992}-1.9 & \cellcolor[rgb]{ .839,  .882,  .945}-23.2 & \cellcolor[rgb]{ .98,  .984,  .996}-0.7 & \cellcolor[rgb]{ .984,  .808,  .82}31.7 & \cellcolor[rgb]{ .988,  .902,  .914}15.4 & \cellcolor[rgb]{ .984,  .792,  .804}34.1 \\
    bond\_europe & \cellcolor[rgb]{ .988,  .98,  .992}1.4 & \cellcolor[rgb]{ .988,  .871,  .882}20.6 & \cellcolor[rgb]{ .988,  .91,  .922}13.8 & \cellcolor[rgb]{ .984,  .733,  .745}44.4 & \cellcolor[rgb]{ .988,  .925,  .937}11.4 & \cellcolor[rgb]{ .973,  .412,  .42}100.0 & \cellcolor[rgb]{ .988,  .875,  .886}20.1 & \cellcolor[rgb]{ .976,  .42,  .427}98.9 & \cellcolor[rgb]{ .976,  .447,  .455}94.2 & \cellcolor[rgb]{ .98,  .624,  .635}63.5 & \cellcolor[rgb]{ .98,  .639,  .651}60.6 & \cellcolor[rgb]{ .984,  .757,  .765}40.6 & \cellcolor[rgb]{ .984,  .737,  .745}44.0 & \cellcolor[rgb]{ .988,  .902,  .914}15.0 & \cellcolor[rgb]{ .988,  .922,  .933}12.0 & \cellcolor[rgb]{ .984,  .835,  .847}27.0 \\
    bond\_us & \cellcolor[rgb]{ .973,  .976,  .992}-2.3 & \cellcolor[rgb]{ .882,  .914,  .961}-16.3 & \cellcolor[rgb]{ .757,  .827,  .918}-35.9 & \cellcolor[rgb]{ .796,  .855,  .933}-29.7 & \cellcolor[rgb]{ .769,  .835,  .922}-34.1 & \cellcolor[rgb]{ .988,  .875,  .886}20.1 & \cellcolor[rgb]{ .973,  .412,  .42}100.0 & \cellcolor[rgb]{ .988,  .902,  .914}15.1 & \cellcolor[rgb]{ .98,  .984,  .996}-0.7 & \cellcolor[rgb]{ .953,  .965,  .988}-5.1 & \cellcolor[rgb]{ .988,  .91,  .918}14.3 & \cellcolor[rgb]{ .984,  .761,  .773}39.5 & \cellcolor[rgb]{ .988,  .863,  .871}22.3 & \cellcolor[rgb]{ .847,  .886,  .949}-22.2 & \cellcolor[rgb]{ .941,  .953,  .98}-7.3 & \cellcolor[rgb]{ .875,  .906,  .957}-17.6 \\
    bond\_eur & \cellcolor[rgb]{ .988,  .976,  .988}2.2 & \cellcolor[rgb]{ .988,  .871,  .878}21.1 & \cellcolor[rgb]{ .988,  .902,  .914}15.4 & \cellcolor[rgb]{ .984,  .718,  .725}47.4 & \cellcolor[rgb]{ .988,  .918,  .929}12.7 & \cellcolor[rgb]{ .976,  .42,  .427}98.9 & \cellcolor[rgb]{ .988,  .902,  .914}15.1 & \cellcolor[rgb]{ .973,  .412,  .42}100.0 & \cellcolor[rgb]{ .976,  .431,  .443}96.6 & \cellcolor[rgb]{ .98,  .671,  .682}55.3 & \cellcolor[rgb]{ .98,  .651,  .659}58.9 & \cellcolor[rgb]{ .984,  .773,  .784}37.7 & \cellcolor[rgb]{ .984,  .749,  .761}41.6 & \cellcolor[rgb]{ .988,  .906,  .914}14.9 & \cellcolor[rgb]{ .988,  .922,  .933}11.8 & \cellcolor[rgb]{ .984,  .827,  .839}28.0 \\
    eur   & \cellcolor[rgb]{ .988,  .973,  .984}3.1 & \cellcolor[rgb]{ .988,  .867,  .875}21.6 & \cellcolor[rgb]{ .988,  .878,  .89}19.5 & \cellcolor[rgb]{ .98,  .686,  .698}52.4 & \cellcolor[rgb]{ .988,  .894,  .906}16.5 & \cellcolor[rgb]{ .976,  .447,  .455}94.2 & \cellcolor[rgb]{ .98,  .984,  .996}-0.7 & \cellcolor[rgb]{ .976,  .431,  .443}96.6 & \cellcolor[rgb]{ .973,  .412,  .42}100.0 & \cellcolor[rgb]{ .98,  .647,  .655}59.7 & \cellcolor[rgb]{ .98,  .639,  .647}60.8 & \cellcolor[rgb]{ .984,  .804,  .816}32.3 & \cellcolor[rgb]{ .984,  .769,  .776}38.6 & \cellcolor[rgb]{ .988,  .882,  .89}19.0 & \cellcolor[rgb]{ .988,  .91,  .922}13.9 & \cellcolor[rgb]{ .984,  .808,  .82}31.6 \\
    gbp   & \cellcolor[rgb]{ .988,  .988,  1}0.5 & \cellcolor[rgb]{ .984,  .839,  .851}25.9 & \cellcolor[rgb]{ .988,  .863,  .875}22.0 & \cellcolor[rgb]{ .984,  .729,  .741}45.3 & \cellcolor[rgb]{ .988,  .875,  .886}20.0 & \cellcolor[rgb]{ .98,  .624,  .635}63.5 & \cellcolor[rgb]{ .953,  .965,  .988}-5.1 & \cellcolor[rgb]{ .98,  .671,  .682}55.3 & \cellcolor[rgb]{ .98,  .647,  .655}59.7 & \cellcolor[rgb]{ .973,  .412,  .42}100.0 & \cellcolor[rgb]{ .984,  .78,  .792}36.1 & \cellcolor[rgb]{ .988,  .906,  .918}14.7 & \cellcolor[rgb]{ .984,  .839,  .851}26.1 & \cellcolor[rgb]{ .988,  .859,  .871}22.9 & \cellcolor[rgb]{ .988,  .918,  .925}12.9 & \cellcolor[rgb]{ .984,  .835,  .847}26.9 \\
    chf   & \cellcolor[rgb]{ .988,  .976,  .988}2.5 & \cellcolor[rgb]{ .988,  .941,  .953}8.2 & \cellcolor[rgb]{ .988,  .988,  1}0.2 & \cellcolor[rgb]{ .988,  .863,  .871}22.4 & \cellcolor[rgb]{ .973,  .976,  .992}-1.9 & \cellcolor[rgb]{ .98,  .639,  .651}60.6 & \cellcolor[rgb]{ .988,  .91,  .918}14.3 & \cellcolor[rgb]{ .98,  .651,  .659}58.9 & \cellcolor[rgb]{ .98,  .639,  .647}60.8 & \cellcolor[rgb]{ .984,  .78,  .792}36.1 & \cellcolor[rgb]{ .973,  .412,  .42}100.0 & \cellcolor[rgb]{ .984,  .776,  .784}37.3 & \cellcolor[rgb]{ .984,  .773,  .78}37.9 & \cellcolor[rgb]{ .988,  .953,  .965}6.7 & \cellcolor[rgb]{ .988,  .945,  .957}7.9 & \cellcolor[rgb]{ .988,  .871,  .878}21.0 \\
    jpy   & \cellcolor[rgb]{ .98,  .98,  .996}-1.2 & \cellcolor[rgb]{ .875,  .91,  .961}-17.3 & \cellcolor[rgb]{ .839,  .882,  .945}-23.0 & \cellcolor[rgb]{ .875,  .91,  .961}-17.5 & \cellcolor[rgb]{ .839,  .882,  .945}-23.2 & \cellcolor[rgb]{ .984,  .757,  .765}40.6 & \cellcolor[rgb]{ .984,  .761,  .773}39.5 & \cellcolor[rgb]{ .984,  .773,  .784}37.7 & \cellcolor[rgb]{ .984,  .804,  .816}32.3 & \cellcolor[rgb]{ .988,  .906,  .918}14.7 & \cellcolor[rgb]{ .984,  .776,  .784}37.3 & \cellcolor[rgb]{ .973,  .412,  .42}100.0 & \cellcolor[rgb]{ .984,  .757,  .769}40.5 & \cellcolor[rgb]{ .941,  .953,  .98}-7.1 & \cellcolor[rgb]{ .988,  .976,  .988}2.5 & \cellcolor[rgb]{ .965,  .973,  .992}-3.3 \\
    gold  & \cellcolor[rgb]{ .984,  .984,  .996}-0.1 & \cellcolor[rgb]{ .988,  .91,  .922}13.8 & \cellcolor[rgb]{ .984,  .984,  .996}-0.5 & \cellcolor[rgb]{ .988,  .933,  .945}9.8 & \cellcolor[rgb]{ .98,  .984,  .996}-0.7 & \cellcolor[rgb]{ .984,  .737,  .745}44.0 & \cellcolor[rgb]{ .988,  .863,  .871}22.3 & \cellcolor[rgb]{ .984,  .749,  .761}41.6 & \cellcolor[rgb]{ .984,  .769,  .776}38.6 & \cellcolor[rgb]{ .984,  .839,  .851}26.1 & \cellcolor[rgb]{ .984,  .773,  .78}37.9 & \cellcolor[rgb]{ .984,  .757,  .769}40.5 & \cellcolor[rgb]{ .973,  .412,  .42}100.0 & \cellcolor[rgb]{ .988,  .898,  .91}15.7 & \cellcolor[rgb]{ .988,  .906,  .918}14.4 & \cellcolor[rgb]{ .984,  .804,  .812}32.6 \\
    wti   & \cellcolor[rgb]{ .988,  .984,  .996}1.0 & \cellcolor[rgb]{ .984,  .804,  .816}32.3 & \cellcolor[rgb]{ .984,  .776,  .788}36.9 & \cellcolor[rgb]{ .984,  .788,  .8}34.9 & \cellcolor[rgb]{ .984,  .808,  .82}31.7 & \cellcolor[rgb]{ .988,  .902,  .914}15.0 & \cellcolor[rgb]{ .847,  .886,  .949}-22.2 & \cellcolor[rgb]{ .988,  .906,  .914}14.9 & \cellcolor[rgb]{ .988,  .882,  .89}19.0 & \cellcolor[rgb]{ .988,  .859,  .871}22.9 & \cellcolor[rgb]{ .988,  .953,  .965}6.7 & \cellcolor[rgb]{ .941,  .953,  .98}-7.1 & \cellcolor[rgb]{ .988,  .898,  .91}15.7 & \cellcolor[rgb]{ .973,  .412,  .42}100.0 & \cellcolor[rgb]{ .988,  .882,  .894}18.9 & \cellcolor[rgb]{ .984,  .78,  .788}36.6 \\
    grain & \cellcolor[rgb]{ .988,  .973,  .984}3.3 & \cellcolor[rgb]{ .988,  .894,  .906}16.4 & \cellcolor[rgb]{ .988,  .894,  .906}16.4 & \cellcolor[rgb]{ .988,  .894,  .906}16.9 & \cellcolor[rgb]{ .988,  .902,  .914}15.4 & \cellcolor[rgb]{ .988,  .922,  .933}12.0 & \cellcolor[rgb]{ .941,  .953,  .98}-7.3 & \cellcolor[rgb]{ .988,  .922,  .933}11.8 & \cellcolor[rgb]{ .988,  .91,  .922}13.9 & \cellcolor[rgb]{ .988,  .918,  .925}12.9 & \cellcolor[rgb]{ .988,  .945,  .957}7.9 & \cellcolor[rgb]{ .988,  .976,  .988}2.5 & \cellcolor[rgb]{ .988,  .906,  .918}14.4 & \cellcolor[rgb]{ .988,  .882,  .894}18.9 & \cellcolor[rgb]{ .973,  .412,  .42}100.0 & \cellcolor[rgb]{ .988,  .867,  .878}21.5 \\
    metal & \cellcolor[rgb]{ .988,  .973,  .984}2.8 & \cellcolor[rgb]{ .984,  .733,  .741}44.8 & \cellcolor[rgb]{ .984,  .784,  .792}35.8 & \cellcolor[rgb]{ .984,  .706,  .718}49.1 & \cellcolor[rgb]{ .984,  .792,  .804}34.1 & \cellcolor[rgb]{ .984,  .835,  .847}27.0 & \cellcolor[rgb]{ .875,  .906,  .957}-17.6 & \cellcolor[rgb]{ .984,  .827,  .839}28.0 & \cellcolor[rgb]{ .984,  .808,  .82}31.6 & \cellcolor[rgb]{ .984,  .835,  .847}26.9 & \cellcolor[rgb]{ .988,  .871,  .878}21.0 & \cellcolor[rgb]{ .965,  .973,  .992}-3.3 & \cellcolor[rgb]{ .984,  .804,  .812}32.6 & \cellcolor[rgb]{ .984,  .78,  .788}36.6 & \cellcolor[rgb]{ .988,  .867,  .878}21.5 & \cellcolor[rgb]{ .973,  .412,  .42}100.0 \\
    \end{tabular}%
	         }
\end{table}%

\bigskip
The time it takes to calibrate the entire Bates model is of about 5 hours: the same argumentation holds here about the need to invert the characteristic function, but the additional computational time with respect to Heston is due to the increasing number of parameters that expand the dimensionality of the calibration problem.

