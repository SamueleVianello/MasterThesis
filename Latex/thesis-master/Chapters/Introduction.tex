\chapter{Introduction}
\label{chpr:intro}

\bigskip


Bitcoin was presented in \citep{BTC2008} as a peer-to-peer protocol for electronic cash that would allow online payments to be sent directly from one party to another without going through a financial institution.
It started off in 2009 as system that was only used by a niche of people in online cryptology forums  to experiment with the transaction protocol. It took a few years for Bitcoin to gain public notoriety, while the price kept rising and having quick crashes. 

In particular, Bitcoin generated a lot of buzz in 2017, a year that registered the price increase from 998 \$ to 13,412 \$ in January 2018 with an all-time high of 19,666 \$ on the 17th of December. 
Like during any of history's gold rushes, many people joined the trend and tried to jump on the train  of quick and easy money and were let down when the price deflated to a level of six thousand dollars in 2018 and has lately stabilized around three thousands.

Setting aside the mere numerical value of the price, Bitcoin was the first protocol to solve the problem of double-spending without the need for a centralized party: bitcoins can be transferred but not duplicated, as they only exist as validated transaction in the distributed blockchain. 
These features allow Bitcoin to have a chance at becoming a global, instantaneous and free payment network that would make wealth transfer as easy as online data sharing.
In the same way that e-mail substituted post mail, Wikipedia and other knowledge-based website outdated paper encyclopaedias, music and film streaming services are becoming the new user friendly experience for the two industries, Bitcoin presents itself as a system to exchange wealth between users without the need for banks or other trusted third parties.

Furthermore, Bitcoin is the first digital currency to achieve scarcity in the digital realm: its monetary policy based on deterministic supply mimics the progressive scarcity of gold. It is this \textit{deterministic} supply that make Bitcoin a suitable long term investment, since it prevents arbitrary increase in its total available amount, unlike what happens with \textit{fiat} money.

\bigskip
For these reasons we believe that Bitcoin can be  \textit{digital gold} with an embedded secure network and its characteristics make it resemble more closely a crypto-asset rather than a crypto-currency.
Even though there are a number of supporters of this idea, for instance in \citep{DYHRBERG2016} it is shown that Bitcoin possesses the same hedging abilities of gold, there is yet no general consensus on the matter and different studies get to the opposite result: see for example \citep{KLEIN2018}.

Bitcoin has been called many names: a bubble, a Ponzi scheme, but also defined as sound money and store of value. We agree with the latter and believe that if Bitcoin is  true digital gold, then its value will express the huge potential that has been so far limited by scepticism and misunderstandings.

\bigskip

In the present work, we intend to first study the correlation that exists between Bitcoin and other types of standard assets, both by analysing the empirical correlation of the returns with its significance and by calibrating more sophisticated models such as \textit{jump diffusion} and \textit{stochastic volatility} models.
Secondly, we want to explore the diversification property of adding Bitcoin to a portfolio of asset, by computing the optimal allocation for different levels of risk and expected return in a Markowitz mean-variance framework and by optimizing on the CVaR as the portfolio risk measure.

\bigskip

The inclusion of Bitcoin in an investment portfolio is strongly suggested in \citep{hedge_safe_haven} where the authors arrive to the conclusion that the digital currency is indeed a great diversifier, while not performing as well as a hedge or safe haven. In \citep{andrianto} the diversification properties of Bitcoin are studied by inserting it into a portfolio composed of 8 assets including stocks, currencies and commodities and performing a Markowitz mean-variance analysis.
Finally in \citep{caveat_emptor} the authors study the performance of a portfolio that includes Bitcoin  adopting a CVaR framework: the result is that a small percentage of wealth should be invested in the digital asset.
In the last part of our thesis, we aim to update the results of the papers we just referenced by including the most recent price data, considering a total of 16 assets (15 standard assets plus Bitcoin) and performing a comparison between the two asset allocation frameworks.
Our final results match the outcomes of those articles.

\bigskip
\noindent
The present thesis has been written during the author's fellowship at the Digital Gold Institute(\href{https://www.dgi.io/}{https://www.dgi.io/}), a research and development center focused on teaching, consulting, and advising about scarcity in the digital domain (Bitcoin and crypto-assets) and the underlying blockchain technology.

\bigskip

\section{Thesis structure}
Our work is structured into 5 chapters.

\bigskip
\noindent
In Chapter \ref{chpr:corr_analysis} we study the empirical correlation between the assets taken into considerations (Section \ref{emp_corr}). We also take a look at the significance of each correlation value by performing two statistical tests, Pearson's \textit{t}-test and a permutation test, in Section \ref{sec:corr_significance}. Finally, Section \ref{sec:rolling_cor} investigates the rolling correlation between Bitcoin and the other assets.

\bigskip
\noindent
In Chapter \ref{chpr:models} we present each asset model that we are going to adopt in our work. We first give a brief overview of the building blocks necessary to understand the mathematical framework of the models: definitions of geometric Brownian motions, compound Poisson processes and CIR processes are all presented in Section \ref{sec:notions}. 
In the following sections we consider the models themselves. We  present the univariate formulations of Merton (Section \ref{sec:merton}), Heston (Section \ref{sec:heston}) and Bates (Section \ref{sec:bates}) and provide a possible generalization to the multi-asset case following the parsimonious approach as reported in \citep{PARSIMONIOUS2011}

\bigskip
\noindent
In Chapter \ref{chpr:calibration} we explain how to calibrate each different model: first we present a general overview of the maximum likelihood estimation (Section \ref{sec:calib_overview}) and then we proceed to specify this approach for our three models in Sections \ref{sec:merton_cal} , \ref{sec:heston_cal} and \ref{sec:bates_cal}. 
The last section of the chapter is devoted to the details of our implementation and to the presentation of the numerical results (Section \ref{sec:results_cal}).

\bigskip
\noindent
Chapter \ref{chpr:markowitz} is where we study the advantages of including Bitcoin in our portfolio. In Section \ref{sec:markowitz_theory} we present the general framework for the \textit{modern portfolio theory} by Markowitz. The following Section \ref{sec:markowitz_frontier} shows how to plot the efficient frontier and the results for the optimal allocation problem.
In Section \ref{sec:cvar_theory} we change the portfolio risk measure to the daily CVaR and in Section \ref{sec:cvar_frontier} the efficient frontier and the allocations with the CVaR are presented with a comparison to those of Markowitz.

\bigskip
\noindent
Chapter \ref{chpr:conclusion} concludes our work and sums up the main results. It also includes some final remarks to this study.

\bigskip

\section{Dataset  Presentation}
\bigskip
The dataset on which we focus our study contains 2163 observations of the prices of 16 assets valued daily (excluding holidays and weekends) from the 19th of July 2010 till the 2nd of November 2018 (all data provided by Bloomberg).

The assets we included in our analysis are grouped into five different classes , as explained in the following list (in brackets we indicate the shortened name that is used in the tables and graphs):

\begin{enumerate}
	\item Bitcoin (btc): Value of a single bitcoin, quoted in dollars.
	\bigskip
	\item Stock indexes:
	\begin{itemize}
		\item S\&P500 (sp500): American stock market index based on 500 large company with stock listed either on the NYSE or NASDAQ.
		\item EURO STOXX 50 (eurostoxx): equity index of eurozone stocks, covering 50 stocks from 11 eurozone countries.
		\item MSCI BRIC (bric): market cap weighted index designed to measure the equity market performance across the emerging country indexes of Brazil, Russia, India and China.
		\item NASDAQ(nasdaq): market cap weighted index including all NASDAQ tiers: Global Select, Global Market and Capital Market.
	\end{itemize}
	\item Bond indexes:
	\begin{itemize}
		\item BBG Pan European (bond\_europe): Bloomberg Barclays Pan-European Aggregate Index that tracks fixed-rate, investment-grade securities issued in different European currencies.
		\item BBG Pan US (bond\_us): Bloomberg Barclays US Aggregate Bond Index, a benchmark that measures investment grade, US dollar-denominated, fixed-rate taxable bond market.
		\item BBG Pan EurAgg (bond\_eur): similar to the Pan European but it only considers securities issued in Euros.
	\end{itemize} 
	\item Currencies:
	\begin{itemize}
		\item EUR/USD (eur): spot value of one Euro in  US dollars. 
		\item GBP/USD (gbp): spot value of one British Pound in US dollars.
		\item CHF/USD (chf): spot value of one Swiss Franc in  US dollars.
		\item JPY/USD (jpy): spot value of one Japanese Yen in  US dollars.
	\end{itemize}
	\item Commodities:
	\begin{itemize}
		\item Gold (gold): price of gold measured in USD/Oz.
		\item WTI (wti): price of crude oil used as benchmark in oil pricing and as the underlying commodity in the NYMEX oil future contracts.
		\item Grain (grain): S\&P GPSCI index that measures the performance of the grain commodity market.
		\item Metals (metal): S\&P GSCI Industrial Metals index that measures the movements of industrial metal prices including aluminium, copper, zinc, nickel and lead.
	\end{itemize}
\end{enumerate}

