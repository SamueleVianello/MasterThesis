\chapter{Presentation of the Models}
\label{chpr:models}
In this chapter we will present the stochastic frameworks in which we developed our analysis. We first introduce a \textit{jump diffusion} (JD) model presented in 1976 by R.C. Merton: he added log-normal jumps to the simple Black\&Scholes dynamics of the asset price. Then we move to the \textit{stochastic volatility} (SV) model of Heston 1993. Heston introduced a new stochastic process that accounts for the variance of the underlying price which evolves as in B\&S  but with a stochastic volatility term.
The last model we will present was introduced by Bates in 1996 and it is the combination of the former two: an asset dynamics which includes jumps and is driven by a stochastic volatility.
All models are first introduced in the one dimensional case and then generalised to the $n$ dimensional case which was then implemented in our code.

\bigskip

\section{Preliminary Notions}
\label{sec:notions}
In this section we will briefly present the equation of a geometric brownian motion, introduce the notion of Poisson process and present the 
CIR process. All of these building blocks will be required to fully understand the models to follow.


\subsection{Geometric Brownian Motion}
The simplest continuous dynamics to describe the price of an asset is that of a geometric Brownian motion:
\begin{equation}
	\label{eq:GBM}
	\frac{dS_t}{S_t} = \mu dt + \sigma dW_t,
\end{equation}

where $S_t$ represents the price of the asset at time $t$, $\mu$ is the (constant) drift and $\sigma$ is the (constant) volatility. $W_t$ is a Wiener process (or Brownian motion).
This is the standard and most widespread stochastic differential equation to model asset dynamics, so we will only present those results that will be later used in our study.


Applying Ito's lemma to the previous equation, we can also explicitly express the dynamics of the log-returns $X_t = log(S_t)$ , obtaining:
\begin{equation}
	dX_t = (\mu - 	\frac{\sigma^2}{2}) dt + \sigma dW_t .
\end{equation}

This stochastic differential has a simple solution which can be computed via stochastic integrals and allows us to describe the dynamics of the log-returns at each instant $t$ starting from $t=0$:
\begin{equation}
	\label{GBM_ret_sol}
	X_t = X_0 + (\mu - 	\frac{\sigma^2}{2}) t + \sigma W_t.
\end{equation}

Thanks to (\ref{GBM_ret_sol}) we can now express the price dynamics of the asset by inverting the relation with the returns: $S_t = e^{X_t}$. 
We thus obtain the solution to (\ref{eq:GBM}):
\begin{equation}
	S_t = S_0 \;e^{(\mu - 	\frac{\sigma^2}{2}) t + \sigma W_t}.
\end{equation}

Given that $S_t = e^{X_t}$ and that $X_t$ by its equation is a generalized brownian motion and hence we have $X_t - X_0 \sim \mathcal{N}(\mu - 	\frac{\sigma^2}{2}, \sigma^2)$, the resulting distribution of prices at time $t$ as units of the initial value is distibuted as a log-normal.

The great success of these framework comes from the simplicity of its dynamics. In particular, since the log-returns follow a Gaussian distribution, $\mu$ and $\sigma$ are easy to calibrate from data and the formulas for pricing options  are often explicit.
As one can imagine, a simple model can only explain simple phenomena: that's why we have such a great deal of \textit{new and improved} version of equation \eqref{eq:GBM}.

\subsection{Poisson Process and Compound Poisson Process}
\label{sub:poisson}
Consider a sequence of \textit{independent} exponential random variables  $\{\tau_i\}_{i\geq1}$ with parameter $\lambda$\footnote{An exponential random variable $\tau$ of parameter $\lambda$ has a cumulative distibution function of the form: $\mathbb{P}(\tau \geq y) = e^{-\lambda y}$} and let $T_n = \sum_{i=1}^{n}\tau_i$. Then we can define the \textit{Poisson process} $N_t$ as
\begin{equation}
 	N_t = \sum_{n\geq 1} \mathds{1}_{t \geq T_n},
\end{equation}

where $\mathds{1}_{condition}$ is 1 if the \textit{condition} is true, 0 if it is false.

$N_t$ is thus a piece-wise constant RCLL\footnote{RCLL is shorthand for right continuous with left limit.} process with jumps that happen at times $T_n$ and are all of size 1, as we can see from Figure \ref{fig:pois}.
An important property of Posson processes is that they have independent and stationary increments, meaning that the increment of $N_t - N_s$ (with $s\leq t$) is independent from the history of the process up to $N_s$ and has the same law of $N_{t-s}$.
At any time $t$, $N_t$ is distributed as a Poisson of parameter $\lambda t$, which means it is a discrete random variable on the integer set with
\begin{equation}
\label{eq:pois_pdf}
\mathbb{P}( N_t = n) = e^{-\lambda t}\frac{(\lambda t)^n}{n!}.
\end{equation}



\begin{figure}
	\centering
	\begin{subfigure}{.5\textwidth}
		\centering
		\includegraphics[width=\linewidth]{Images/poisson_process.pdf}
		\caption{Poisson processes.}
		\label{fig:pois}
	\end{subfigure}%
	\begin{subfigure}{.5\textwidth}
		\centering
		\includegraphics[width=\linewidth]{Images/compound_poisson_process.pdf}
		\caption{Compound with Gaussian jumps.}
		%$\mathcal{N}(1,1)$
		\label{fig:compound_pois}
	\end{subfigure}
	\caption[Trajectories of Poisson processes]{Poisson processes and compound Poisson processes with different $\lambda$.}
	\label{fig:pois_process}
\end{figure}

When working with jump diffusion processes, it is often the case that there is no explicit formula for its density, thus  we usually resort to characteristic functions. The characteristic function of $N_t$ is given by 
\begin{equation}
\label{eq:pois_chf}
	\phi_{N_t}(u)=e^{\lambda t (e^{iu}-1)}
\end{equation}

The computations to get \eqref{eq:pois_chf} from \eqref{eq:pois_pdf} are carried out in Appendix \ref{app:A}. 

\bigskip
For financial applications, it is of little interest to have a process with a single possible jump size. The \textit{compound} Poisson processes are a generalization of Poisson processes  where  the jump sizes can have an arbitrary distribution. More precisely, consider a Poisson process $N_t$ with parameter $\lambda$ and a sequence of i.i.d\footnote{i.i.d stands for independent and identically distributed.} variables $\{Y_i\}_{i\geq 1}$ with law $f_Y(y)$. Then the process defined by
\begin{equation}
	X_t = \sum_{n=1}^{N_t} Y_i
\end{equation}
is a compound Poisson process. Examples of this kind of process are plotted in Figure \ref{fig:compound_pois}.

As before, we developed the computations to obtain an expression for the characteristic function of $X_t$ in Appendix \ref{app:A}. The resulting expression depends on the distribution of $Y$, specifically from its characteristic function $\phi_Y(u)$:

\begin{equation}
	\phi_{X_t}(u) = e^{\lambda t (\phi_Y(u)-1)}
\end{equation}

Both in Merton's and in the Heston's models there will be a jump component driven by a compound Poisson process with Gaussian jump sizes, as we will see in the following paragraphs.

\subsection{CIR Process}
The CIR process was introduced in \citep{CIR85} in 1985 by Cox, Ingersoll and Ross (hence the name CIR) as a generalization of a Vasicek process \citep{VASICEK77} to model the mean reverting dynamics of interest rates.
Following their notation, the differential equation for the  evolution of the rate is given by:
\begin{equation}
\label{eq:cir}
	dr_t = \kappa(\theta - r_t) dt + \sigma \sqrt{r_t}dW_t
\end{equation}
where we have three parameters that characterize it: $\theta$ is the \textit{long-term value} of the rate, the asymptotic level which it tends to settle at in the long run;$\kappa$ is the \textit{mean-reversion rate}, the speed at which the rate is pulled back to the $\theta$ value; finally $\sigma$ accounts for the \textit{volatility} of the stochastic component.
When $\kappa,\theta >0$, equation (\ref{eq:cir}) represents a first order mean-reverting auto-regressive process. Moreover, as reported in \citep{HESTONTRAP}, we know that the process will not hit zero if the following condition is satisfied:
\begin{equation}
	2\kappa\theta > \sigma^2.
\end{equation}
This condition is usually referred to as \textit{Feller} condition, as stated in the referenced paper.

An example of a CIR process is shown in Figure \ref{fig:cir_proc}, where the mean-reverting effect is  clearly visible.

\begin{figure}
	\centering
	\includegraphics[width=0.6\textwidth]{Images/cir_process.pdf}
	\caption[Trajectories of two CIR processes]{Two trajectories of the same CIR process: $dr_t = 3(2 - r_t) dt + 0.5 \sqrt{r_t}dW_t$ with different starting point. We can see the mean-reverting effect that attracts both trajectories to the value $\theta=2$. }
	\label{fig:cir_proc}
\end{figure}

When modelling interest rate in continuous time, having $r_t$ hit zero is not an issue since when $r_t=0$ equation (\ref{eq:cir}) reduces to $dr_t = \kappa\theta dt$ which immediately brings the level back to positive values and hence the square root in the dynamics never loses meaning. Conversely, considering a \textit{discretized} version of (\ref{eq:cir}), as is the case in a simulation framework, one needs to pay attention on how he models the increments since a simple Euler discretization scheme may cause $r_t$ to reach \textit{negative} values\footnote{This model was introduced having in mind the certainty that interest rates could never be negative, hence the introduction of a square root in the dynamics. Given recent years interest rates levels, this in no more the case.} and invalidate the whole model representation.

The non negativity of the CIR process will become fundamental when we introduce Heston and Bates SV models, where the (stochastic) variance of the process driving the asset price will be model as a CIR process.

Since it will be useful later on in the paper, we also present the \textit{stationary} distribution of $r_t$. Due to the mean reversion effect, $r_t$ will approach a \textit{gamma} distribution with density:
\begin{equation}
\label{eq:cir_pdf}
f_{r_\infty}(x) = \frac{\omega^\nu}{\Gamma(\nu)} x^{\nu-1}e^{-\omega x},
\end{equation}
where
\begin{equation*}
\omega= \frac{2\kappa}{\sigma^2}\: , \: \nu= \frac{2\kappa\theta}{\sigma^2}.
\end{equation*}

Its \textit{moment generating function}, which will be useful later as well, is defined as follows:
\begin{equation}
\label{eq:cir_mgf}
	M_{r_\infty}(z) = \Big(\frac{\omega}{\omega-z}\Big)^\nu .
\end{equation}

\bigskip
\section{Merton Model}
\label{sec:merton}
Let us see now how the different building blocks that we just presented can be combined together by first taking a look at the jump diffusion model by Merton.

%%% ORIGINAL MODEL
\subsection{Original Univariate Model}
The jump diffusion model we are interested in was originally introduced in \citep{MERTON1976} in order to account for the leptokurtic distribution of real market returns and to model sudden falls (or rises) in prices due to the arrival of new information.
The asset price dynamics $S_t$ is modelled as a GBM to which a jump component driven by a compound Poisson process is added:

\begin{equation}
\label{eq:merton_model}
\frac{dS_t}{S_t} = (\mu - \lambda \mu_J) dt + \sigma dW_t  + Y_t dN,
\end{equation}

where $\mu$ and $\sigma$ are respectively the drift and the diffusion of the continuous part, $Y_t$ is a process modelling the intensity of the jumps and $N(t)$ is the Poisson process driving the arrival of the jumps and has parameter $\lambda$.
The jump intensity $Y_t$ is distributed as a log-normal with parameters $(\mu_J, \sigma_J^2)$. This means that $ y_t = \log(Y_t) \sim \mathcal{N}(\mu_J, \sigma_J^2)$

We can rewrite (\ref{eq:merton_model}) in terms of the log-returns $X_t = log(S_t)$ and obtain, following the computations in \citep{MARTIN2007} and using theory from \citep{TANKOV2015}:
\begin{equation}
dX_t = (\mu - \frac{\sigma^2}{2} - \lambda \mu_J)dt + \sigma dW_t + \log(Y_t),
\end{equation}

that has as solution:
\begin{equation}
\label{merton_returns}
X_t =X_0 +  \mu t + \sigma W_t + \sum_{k=1}^{N(t)} y_k,
\end{equation}

where $X_0$ is the initial value of the log-returns, $y_k= \log(Y_k) = \log(Y_{t_k})$ and $t_k$ is the time when the $k^{th}$ Poisson shock from $N(t)$ happens. 
Following Merton in \citep{MERTON1976}, we take $y_k$ \textit{i.i.d.} (independent and identically distributed) and Gaussian, as we already stated.
Another choice for the distribution of $y$ is given in \citep{KOU2002}, where an asymmetric distribution is used to account for positive and negative jump in two different ways. 

%% AGGIUNGERE DENSITA' NEL caso generale [0,T] ?

It is often useful when dealing with market data that are by nature discrete, to consider a \textit{discretized} version of (\ref{merton_returns}) in which the values are sampled at intervals of $\Delta t$ in $[0, T]$. We thus get that for $X_i = \log(\frac{S_{i+1}}{S_i})$:

\begin{equation}
\label{discrete_returns}
X_i =  \mu \Delta t + \sigma \sqrt{\Delta t} \; z +  \sum_{k=1}^{N_{i+1} - N_i} Y_k
\end{equation}

where we denote $X_i = X_{t_i}$, $N_i = N(t_i)$ and $t_i = i \Delta t$ with $i= 0 \dots N$, $t_N = N \Delta t= T$,  $z$ is distributed as a standard Gaussian $ z\sim \mathcal{N}(0,1)$.

The Poisson process $N(t)$ in (\ref{discrete_returns}) is computed at times $t_{i+1}$ and $t_i$ and these quantities are subtracted. Following the results we presented in Section \ref{sub:poisson}, $N_{i+1} - N_i$ is distributed as a Poisson random variable $N$ of parameter $\lambda \Delta t$.
This allows us to provide an explicit formulation for the transition density of the log-returns using the theorem of total probability:

\begin{equation}
\label{transitional}
f_{\Delta x} (x) = \sum_{k=0}^{\infty} \mathbb{P}(N = k) f_{\Delta x | N = k}(x) .
\end{equation}

This equation is a good analytical result but it is not so practical for calibration purposes, as we will explain more in depth in Chapter \ref{chpr:calibration}.


%%% MULTIVARIATE MODEL
\subsection{Multivariate Model}
Starting from the univariate model introduced in \citep{MERTON1976}, we can develop a generalization to $n$ assets including only idiosyncratic jumps:

\begin{equation}
\frac{dS_t^{(j)}}{S_t^{(j)}} = (\mu_j - \lambda_j \mu_{J_j}) dt + \sigma_j dW_t^{(j)} + Y^{(j)}_t dN^{(j)}_t
\end{equation}

where $\mathbf{S}_t$ are the prices of the assets, $j = 1, ...  ,n$ represents the asset, $\mu_j$ are the drifts, $\sigma_j$ are the diffusion coefficients, $W^{(j)}_t$ are the components of an $n$-dimensional Wiener process $ \mathbf{W}_t$ with $dW^{(i)}dW^{(j)}=\rho_{i,j}$, $Y_j$ represent the intensities of the jumps and  $y^{(j)} = \log(Y^{(j)})$ are distributed as Gaussian: $y^{(j)}  \sim \mathcal{N}(\mu_{J_j} , \sigma_{J_j}^2)$. Finally, $N^{(j)}(t)$ are Poisson processes with parameters $\lambda_j$, which are independent of $\mathbf{W}_t$ and of one another. 

\bigskip

\section{Heston Model}
\label{sec:heston}

The Heston model was presented in 1993 in \citep{HESTON93} as a new framework to model stochastic volatility in the asset price dynamics, which allows to better fit the skewness and kurtosis of the log-return distribution and account for the changes in the volatility of the price process through time.

\subsection{Univariate Heston Model}
The Heston model belongs to the family of SV processes: generalizations of B\&S model in which the volatility is no more constant but is itself  stochastic.

Starting with a GBM as in the B\&S framework, we obtain the dynamics of the price process simply by allowing the volatility in (\ref{eq:GBM}) to evolve over time and specifying how this evolution takes place.
In particular, the dynamics of the \textit{instantaneous} variance $V_t = \sigma_t^2$ for the Heston model is described by a CIR process :
\begin{equation}
\label{eq:heston_price}
	\frac{dS_t}{S_t} = \mu dt +\sqrt{ V_t} dW_t^S,
\end{equation}
\begin{equation}
\label{eq:heston_variance}
	dV_t = \kappa (\theta - V_t) dt + \sigma_V \sqrt{V_t} dW_t^V,
\end{equation}

where $\mu$ represents the drift in the asset prices, $\kappa>0$ and $\theta>0$ are respectively the \textit{mean-reversion} rate  and the long-run level for the $V_t$ process, $\sigma_V>0$ is often referred to as the \textit{volatility of volatility} parameter, in short vol-of-vol.
The two brownian motions $W_t^S$ and $W_t^V$ are correlated with correlation coefficient equal to $\rho$.
It can be proven that the variance process is always non-negative if the \textit{Feller} condition
 \begin{equation}
 \label{eq:feller_condition}
	2\kappa\theta \geq \sigma_V
 \end{equation} 
 is satisfied. If the same inequalities holds strictly, we have that the variance process will always be strictly positive.

Let us now consider the dynamics of the log-return $x_t = \log (S_t)$, as we did in the Merton case. 
Unfortunately, given the increased complexity of the model due to the SV part, an explicit formula for the density of the log-return is not available and it has to be computed from the Fourier transform of the characteristic function:
\begin{equation}
\label{eq:chf_inversion}
f_{x_t}(x) = \frac{1}{2\pi}\int_{-\infty}^{+\infty} \phi_{x_t}(i u) e^{i u x} du .
\end{equation}

as explained in the reference paper by Heston.

In (\ref{eq:chf_inversion}) we omitted the dependence of $f_{x_t}(x)$ and $\phi_{x_t}(u)$ on model parameters and on the initial values of the log-returns $x_0$ and of the variance process $V_0$. 

\bigskip

To derive the expression of the characteristic function $\phi_{x_t}(u)$, one has to solve a couple of \textit{Fokker-Planck} partial differential equation as is shown in the Appendix of the reference paper by Heston \citep{HESTON93}. This procedure is beyond the scope of this thesis and thus we will only report the final result, which is a log-affine equation conditioned on the information available at time $t = 0$:

\begin{subequations}
\begin{align}
\label{eq:heston_chf+ABC}
	\phi_{x_t}(u| x_0, V_0) &= \exp\{A(t,u) + B(t,u) x_0 + C(t,u) V_0\}\nonumber, \\
	A(t,u) &= \mu u i t +  \frac{\kappa\theta}{\sigma_V^2} \bigg( (\kappa - \rho\sigma_V u i +d)t - 2 \log\Big[  \frac{1-ge^{dt}}{1-g} \Big] \bigg),\\
	B(t,u) &= i u, \\
	C(t,u)&= \frac{\kappa - \rho\sigma_V u i +d}{\sigma_V^2} \:\Big[\frac{1-e^{dt}}{1-ge^{dt}}\Big],
\end{align}
\end{subequations} 
where :
\begin{equation*}
\begin{split}
d&=\sqrt{(\rho \sigma_V u i - \kappa)^2 + \sigma_V^2(u i + u^2)},\\
g&= \frac{\kappa - \rho\sigma_V u i + d}{\kappa - \rho\sigma_V u i - d}
\end{split}.
\end{equation*} 



This formulation is the one proposed by Heston in his original paper \citep{HESTON93} but it is shown in \citep{HESTONTRAP}  that it has numerical issues when  pricing Vanilla options using Fourier methods. We will not be pricing any instrument, however we may incur in the same errors when calibrating our model through the techniques explained in Chapter \ref{chpr:calibration}. For this reason, we will be using the alternative formulation presented in \citep{HESTONTRAP}:



\begin{subequations}
% heston trap solution
\begin{align}
\label{eq:heston_chf_trap}
\phi_{x_t}^*(u| x_0, V_0) &= \exp\{A(t,u) + B(t,u) x_0 + C(t,u) V_0\}.\nonumber\\
A(t,u) &= \mu u i t +  \frac{\kappa\theta}{\sigma_V^2} \bigg( (\kappa - \rho\sigma_V u i - d)t - 2 \log\Big[  \frac{1-g^*e^{-dt}}{1-g^*} \Big] \bigg),\\
B(t,u) &= i u ,\\
C(t,u)&= \frac{\kappa - \rho\sigma_V u i - d}{\sigma_V^2} \:\Big[\frac{1-e^{-dt}}{1-g^*e^{-dt}}\Big],
\end{align}
\end{subequations} 
where :
\begin{equation*}
\begin{split}
d&=\sqrt{(\rho \sigma_V u i - \kappa)^2 + \sigma_V^2(u i + u^2)},\\
g^*&= \frac{\kappa - \rho\sigma_V u i - d}{\kappa - \rho\sigma_V u i + d} = \frac{1}{g}.
\end{split}
\end{equation*} 

The only difference is that the signs of the $d$ terms are all flipped: the origin of the two representations for the characteristic function lies in the fact that the complex root $d$ has two possible values and the second value is exactly minus the first value.
A uniform choice between the two possibilities cannot be made over all the complex plane to obtain a continuous function due to the presence of a branch cut in the graph of $\sqrt{z}$. Selecting the sign of $d$ as in $\phi_{x_t}^*$ and $g^*$ avoids the discontinuity affecting our future computations.

For this reason, from now on we will be using the second formulation while leaving out the star * from our notation.

\bigskip


Our characteristic function still depends on the initial values $x_0$ and  $V_0$ and is thus often called \textit{conditional} characteristic function on $x_0$ and $V_0$.
We can easily remove the dependence on $x_0$ by considering the distribution of incremental returns, namely $\Delta x_t = \log (S_t / S_0) = x_t - x_0$
%, but as for $V_0$, we will need to do some more work.

Applying the definition of characteristic function:
\begin{equation*}
	\begin{split}
	\phi_{\Delta x_t}(u|V_0) &= \mathbb{E}\big[e^{i u \Delta x_t} |V_0\big] \\
	&=  \mathbb{E}\big[e^{i u ( x_t - x_0)}|V_0\big]\\
	&= \mathbb{E}\big[e^{i u x_t}|V_0\big] e^{-i u x_0}\\
	&= \phi_{x_t}(u) e^{-i u x_0}\\
	&= \exp\{A(t,u) + (B(t,u) - iu) x_0 + C(t,u) V_0\}
	\end{split}
\end{equation*}

and since $B(t,u)  = i u$, the second term in the exponential is equal to zero and we are left with:
\begin{equation}
\label{eq:chf_V0}
\phi_{\Delta x_t}(u|V_0) =  \exp\{A(t,u) + C(t,u) V_0\}
\end{equation}




This equation is however still dependent on the initial value of the variance process. In a simulation framework, this would not be an issue, since we can define the level of $V_0$ ourselves and then generate all the different scenarios. However, if we need to calibrate the model parameters from asset prices, the market data for the variance process are not available. We will address more in depth how to solve this problem later on in Chapter \ref{chpr:calibration}. As for now, we show that we can obtain an \textit{unconditional} expression for the characteristic function of a Heston process by approximating the distribution of $V_t$ with its stationary distribution.

The idea of integrating out the variance was presented in \citep{DRAGULESCU2002} and here we follow a similar approach.
The computations to obtain an unconditional characteristic function are the following:
\begin{equation*}
\begin{split}
	\phi_{\Delta x_t}(u) &= \mathbb{E}\big[e^{i u \Delta x_t} \big] \\
	&= \mathbb{E}\big[ \mathbb{E}\big[ e^{i u \Delta x_t} | V_0\big] \big]\\
	&=  \int_{0}^{\infty} \mathbb{E}\big[e^{i u ( x_t - x_0)}|V_0 = v \big] f_{V_0}(v) dv \\
	&= \int_{0}^{\infty} \exp\{A(t,u) + C(t,u) v\} f_{V_0}(v) dv \\
	&=  \exp\{A(t,u)\} \int_{0}^{\infty}  \exp \{C(t,u) v\} f_{V_0}(v) dv\\
	&= \exp\{A(t,u) \} M_{V_0}(C(t,u))
\end{split}
\end{equation*}

where we have used the Law of Total Expectation and $M_{V_0}(z)$ indicates the \textit{moment generating function} of $V_0$ as considered stationary, so that of a Gamma distribution. In particular, it will have the same expression as in \eqref{eq:cir_mgf}.



\subsection{Parsimonious Multi-asset Heston Model}
The problem with generalizing the Heston framework to include more than one underlying is that the model becomes quickly very complex: we need 5 parameters for each asset ($\mu, \kappa,\theta, \sigma_V$ and $ \rho$), and we can also model all types of correlation between the different stochastic drivers. In particular, we have 4 types of correlation: 
\begin{itemize}
	\item $ \rho^{S_i , V_i}$:  correlation that we already have in the unidimensional case, that models the way the price and the variance processes of the same asset are linked ;
	\item $ \rho^{S_i , S_j}$: correlation between the movements in prices of two different assets ;
	\item $\rho^{V_i , V_j} $: correlation between the variance processes of two different assets. One might expect that asset of similar nature have a higher correlation both in the price and in the variance processes;
	\item $\rho^{S_i , V_j} $: correlation between the price process of an asset and the variance process of a different one.
\end{itemize}

To overcome this increase in both mathematical and computational complexity, we will use the \textit{parsimonious} model introduced by Szimayer, Dimitroff and Lorenz in \citep{PARSIMONIOUS2011}. Their framework has two main characteristics: each single-asset sub-model forms a traditional Heston model and the parameters are  composed by $n$ single-asset Heston parameter sets and $n (n-1)/2$  \textit{asset-asset} correlations.

The $n$-dimensional model  hence becomes:

%\begin{equation}
%\frac{dS_i(t)}{S_i(t)} = \mu dt +\sqrt{ V_t} dW_t^S
%\end{equation}
%\begin{equation}
%dV_t = \kappa (\theta - V_t) dt + \sigma_V \sqrt{V_t} dW_t^V
%\end{equation}

 \begin{multline}
 \label{eq:mv_heston}
\begin{pmatrix}
	dS_i(t) / S_i(t)\\
	dV_i(t)
\end{pmatrix}
= \begin{pmatrix}
\mu_i\\
\kappa_i (\theta_i - V_i(t))
\end{pmatrix}
dt  +\\ \begin{pmatrix}
\sqrt{V_i(t)} & 0 \\
0 & \sigma_{V_i} \sqrt{V_i(t)} 
\end{pmatrix}
\begin{pmatrix}
1 & 0 \\
 \rho_i & \sqrt{1-\rho_i^2} 
\end{pmatrix}
\begin{pmatrix}
dW^{S_i}(t)\\
dW^{V_i }(t)
\end{pmatrix},
\end{multline}

where $dW^{S_i}(t)$ and $dW^{V_i}(t)$ are independent processes and we have written explicitly the dependence between the price and the variance processes.

To fully characterize the model, we need to describe how the different $W^{S_i}$ and $W^{V_i}$ are correlated. In accordance with what was stated earlier, the correlation structure is the following:
\begin{equation}
\label{eq:heston_cor}
\Sigma^{(S,V)} = cor(\boldsymbol{W}^{S}, \boldsymbol{W}^{V}) = \begin{pmatrix}
\Sigma^{S} & 0 \\
0& I_n
\end{pmatrix},
\end{equation}

in which $\Sigma^S = cor(\boldsymbol{W}^{S})$.
Equation (\ref{eq:heston_cor}) mathematically represents what we defined as the correlation structure for our model: we only explicitly define the \textit{asset-asset} correlations through matrix $\Sigma^S$, the dependence of every variance  on other processes is carried over via $\Sigma^S$ and the correlations $\rho_i$.
More clearly, let $(\Sigma^S)_{i,j} = \rho_{i,j}$\footnote{Of course we will have $\rho_{i,j} = 1 $ whenever $i=j$.}:
\begin{itemize}
	\item $dW^{S_i}(t) dW^{S_j}(t) = \rho_{i,j} dt$
	\item $dW^{S_i}(t) dW^{V_j}(t) = \rho_{i,j} \rho_j dt$
	\item $dW^{V_i}(t) dW^{V_j}(t) = \rho_i \rho_{i,j}\rho_j dt $ if $i\neq j$, $dt$ otherwise
\end{itemize}

A detailed representation of a 2-asset \textit{parsimonious} Heston model can be found in the reference paper \citep{PARSIMONIOUS2011}.

%	\begin{bmatrix}
%	ax_{0} + bx_{1} \\           
%	\vdots \\
%	ax_{n-1}+bx_{n}
%	\end{bmatrix} -
%	\begin{bmatrix}
%	z_{1} \\
%	\vdots \\
%	z_{n}
%	\end{bmatrix}
\bigskip
\section{Bates Model}
\label{sec:bates}
Bates model is a way of combining both of the characteristics of Merton (price jumps) and Heston (stochastic volatility) in a single framework.  It was introduced in 1996 by David Bates, an American professor at University of Iowa.

\subsection{Univariate Model}
In his paper \citep{BATES96}, Bates proposes his model as a way of capturing the leptokurtosis in the distribution of log-differences of the USD/DeutscheMark exchange rate. He suggested to improve the versatility of Heston model by including \textit{log-normal, Poisson driven} jumps in the price process, borrowing this addition from Merton model.

Thus, we will have to deal with a total of 8 parameters: $\mu, \kappa, \theta, \sigma_V$ and $ \rho$ for the stochastic volatility part, and $ \mu_J, \sigma_J$ and $ \lambda$ for the jump component. 


\begin{equation}
\frac{dS_t}{S_t} = (\mu - \lambda \mu_J) dt +\sqrt{ V_t} dW_t^S + Y_t dN 
\end{equation}
\begin{equation}
dV_t = \kappa (\theta - V_t) dt + \sigma_V \sqrt{V_t} dW_t^V
\end{equation}

Of course, as these equations are directly obtained from \eqref{eq:heston_price} and \eqref{eq:heston_variance}, we have that the variance process is strictly positive whenever the Feller condition $2\kappa\theta\geq \sigma_V$ is satisfied. As in \eqref{eq:merton_model}, $Y_t$ is  the jump process and is such that $ y_t = \log(Y_t) \sim \mathcal{N}(\mu_J, \sigma_J^2)$, while $N(t)$ is a Poisson process with parameter $\lambda$.


Unfortunately, for the same reason as in previous section, an explicit formula for the distribution of the log-returns $x_t = log(S_t)$ is not available and thus we have to make use of their characteristic function.
The silver lining however is  that to obtain the expression of   $\phi_{x_t}(u)$ we only have to include an extra additive term to the exponential in \eqref{eq:heston_chf_trap} to account for the jumps:

\begin{subequations}
% heston trap solution
\begin{align}
\label{eq:bates_chf_trap}
\phi_{x_t}^*(u| x_0, V_0) &= \exp\{A(t,u) + B(t,u) x_0 + C(t,u) V_0 + D(t,u)\},\nonumber\\
A(t,u) &= \mu u i t +  \frac{\kappa\theta}{\sigma_V^2} \bigg( (\kappa - \rho\sigma_V u i - d)t - 2 \log\Big[  \frac{1-g^*e^{-dt}}{1-g^*} \Big] \bigg),\\
B(t,u) &= i u ,\\
C(t,u)&= \frac{\kappa - \rho\sigma_V u i - d}{\sigma_V^2} \:\Big[\frac{1-e^{-dt}}{1-g^*e^{-dt}}\Big],\\
D(t,u) &= -\lambda \mu_J u i t + \lambda t  \:\Big[ (1+\mu_J)^{ui} e^{\sigma_J^2(ui/2)(ui-1)}-1 \Big]
\end{align},
\end{subequations} 
where :
\begin{equation*}
\begin{split}
d&=\sqrt{(\rho \sigma_V u i - \kappa)^2 + \sigma_V^2(u i + u^2)},\\
g^*&= \frac{\kappa - \rho\sigma_V u i - d}{\kappa - \rho\sigma_V u i + d} = \frac{1}{g}.
\end{split}
\end{equation*} 

We use the  *  formulation for the same desirable numerical properties as already stated, while leaving out the * symbol from our notation from now on.

We can perform the same computations as in the Heston case to obtain the \textit{unconditional} characteristic function of the log-returns to obtain:

\begin{equation}
\label{eq:bates_uncond_chf}
\phi_{\Delta x_t}(u) =  \exp\{A(t,u) + D(t,u)\} M_{V_0}(C(t,u))
\end{equation}
where $M_{V_0}(z)$ indicates the \textit{moment generating function} of $V_0$ as considered stationary and is distributed as a Gamma distribution. In particular, it will have the same expression as in \eqref{eq:cir_mgf} with $\omega = {2\kappa} / {\sigma_V^2} $
and $\nu= \frac{2\kappa\theta}{\sigma_V^2}$.

\subsection{Parsimonious Multi-asset Bates Model}
We  generalize the single-asset Bates model to a multi-asset model in the same way that we did with the Heston: we only model the asset-asset correlation $\rho^{S_i, S_j}$ for each pair of asset.

The model definition  is similar to \eqref{eq:mv_heston}, with the addition of the jump part for the price processes:

\begin{multline}
\begin{pmatrix}
dS_i(t) / S_i(t)\\
dV_i(t)
\end{pmatrix}
= \begin{pmatrix}
\mu_i - \lambda_i \mu_{J_i}\\
\kappa_i (\theta_i - V_i(t))
\end{pmatrix}
dt  \:+\\ 
\begin{pmatrix}
\sqrt{V_i(t)} & 0 \\
0 & \sigma_{V_i} \sqrt{V_i(t)} 
\end{pmatrix}
\begin{pmatrix}
1 & 0 \\
\rho_i & \sqrt{1-\rho_i^2} 
\end{pmatrix}
\begin{pmatrix}
dW^{S_i}(t)\\
dW^{V_i }(t)
\end{pmatrix}
\: + \\
\begin{pmatrix}
Y_i(t) dN^i_t  \\
0   
\end{pmatrix}.
\end{multline}

$Y_i(t)$ are the jump intensities for each asset, such that $\log Y_i(t) \sim \mathcal{N}(\mu_{J_i}, \sigma_{J_i})$ and $N_i(t)$ are the Poisson processes that drive the arrival of the jumps with parameters $\lambda_i$.
We have explicitly expressed the correlation between the two Brownian motions for each asset in the matrix that multiplies the stochastic differential, so that $W^{S_i}(t)$ and $W^{V_i}(t)$ are independent from each other for all $i$.

The only correlations that are left to specify are the ones that exist between $W^{S_i}$ and $W^{S_j}$. Following the same notation as in the previous section, the complete correlation structure is the following:
\begin{equation}
\label{eq:bates_cor}
\Sigma^{(S,V)} = cor(\boldsymbol{W}^{S}, \boldsymbol{W}^{V}) = \begin{pmatrix}
\Sigma^{S} & 0 \\
0& I_n
\end{pmatrix}.
\end{equation}

