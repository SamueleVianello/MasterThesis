\chapter{Conclusions}
\label{chpr:conclusion}


The aim of this work was to study the properties of Bitcoin as a digital asset.

In Chapter \ref{chpr:corr_analysis} we studied the empirical correlation between the returns of Bitcoin and of 15 other assets, grouped in 4 different classes (stock indexes, bonds indexes, currencies and commodities). We found out that the correlation not only is low, but it also is not significantly different from zero, from a statistical point of view. The same holds for the rolling correlation computed in the time span of our data (mid July 2010 to early November 2018).

Then, in Chapter \ref{chpr:calibration} we obtained the same correlation matrix through the calibration of three continuous asset price models: Merton's jump diffusion, Heston's stochastic volatility and Bates's stochastic volatility with jumps in the price process.
For all three, the resulting correlation structure closely resembles the one computed in Chapter \ref{chpr:corr_analysis}: this indicates that there is no real need for a sophisticated model to obtain a feel for what the level of the correlation of Bitcoin with other assets is.

The fact that returns of Bitcoin are not correlated with those of more standard assets lead us to consider the possibility of including the digital asset in an investment portfolio.
In Chapter \ref{chpr:markowitz} we compare the performances of two portfolios, in a Markowitz framework: the first only includes the standard assets while the second also contains Bitcoin. The difference in terms of increased returns or lowered risk is dramatic. We can see this both by looking at the graph of the efficient frontiers and at the numerical results.
We performed the same analysis implementing the daily CVaR at 95\% as the portfolio risk measure obtaining very similar results.
In general, investing 3\% to 5\% of our wealth in Bitcoin proved to be extremely beneficial.

\bigskip

A possible further study can be to analyse the correlation between the most relevant crypto-currencies in the same way that we did. We expect to find that the correlation levels are pretty high and thus it would not prove to be beneficial to include more than one in an investment portfolio. 
Besides, we believe that the properties, the  infrastructure and the community behind Bitcoin make it a cut above other crypto-currencies.

\bigskip



%
%The aim of this thesis is to give an introduction to the Schnorr signature algorithm, starting from the mathematics and the cryptography behind the scheme, and present some of its amazing applications to Bitcoin, detailing the benefits and the improvements that would arise from its deployment. We started with a brief but thorough description of the mathematical structures (Chapter \ref{chpr:math}) and cryptographic primitives (Chapter \ref{chpr:ecc}) that underpin digital signature schemes based on elliptic curve cryptography. In Chapter \ref{chpr:dss} we presented both ECDSA and Schnorr algorithm, respectively the one actually implemented in Bitcoin and the one that is under development. We compared the two schemes, investigating ECDSA lacks and Schnorr benefits, that ranged from security to efficiency. In particular we focused on the linearity property, that turned out to be the key for the higher level construction presented in Chapter \ref{chpr:application}.
%\\
%We have seen how to traduce utilities already implemented in Bitcoin in terms of Schnorr signatures: multi-signature schemes are implemented through MuSig (Section \ref{musig}), whose main advantage is to recover key aggregation; threshold signatures can be deployed through the protocols presented in Section \ref{threshold}, that makes them indistinguishable from a single signature; the last application we studied has been adaptor signature and its benefits to cross-chain atomic swaps and to the Lightning Network.
%
%\bigskip
%\noindent
%The immediate benefits that Schnorr would bring to Bitcoin are improved efficiency (smaller signatures, batch validation, cross-input aggregation) and privacy (multi-signatures and threshold signatures would be indistinguishable from a single signature), leading also to an enhancement in fungibility. All this applications would be possible in a straightforward way after the introduction of Schnorr, that could be brought to Bitcoin through a soft-fork\footnote{Improvements in the protocol have to be made without consensus split.}: the fact that Schnorr is superior to ECDSA in every aspect hopefully will ease the process.
%
%\bigskip
%\noindent
%The last thing we would like to point out is that, by no means, the applications presented in the present work are the unique benefits that Schnorr could bring to Bitcoin. More complex ideas take the names of Taproot \cite{Taproot} and Graftroot \cite{Graftroot}, and are built on top of the concepts of MAST and Pay-to-Contract: through these constructions it would be possible, in the cooperative case, to hide completely the redeem script, presenting a single signature (no matter how complex the script is). For how soft forks need to be implemented after SegWit (i.e. with an upgrade of the version number), there is incentive to develop as many innovations as possible altogether (the presence of too many version numbers with little differences would constitute a lack of privacy): for this reason, it is probable that Schnorr will come to life accompanied by Taproot. 
%\\
%Hopefully, we have convinced the reader that Schnorr (and Bitcoin!) is worth being studied, providing also the tools to properly understand further features and innovations other than the ones presented. Moreover, we hope that you are now motivated not only to delve deeper in the technical side of Bitcoin, but also to approach it from other sides, to fully appreciate its disruptiveness and make yourself an idea of what Bitcoin is and which possibilities it hides.

